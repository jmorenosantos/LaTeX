\documentclass[11pt]{article}
\usepackage{amsmath, amssymb, amsfonts, gensymb, graphicx, enumerate, float, wrapfig, hyperref}
\usepackage[margin=0.5in]{geometry}
\graphicspath{{./}}
\newcommand*{\vs}{\vspace{1cm}}
\newcommand*{\next}{\noindent}
\newcommand*{\set}{\setcounter{equation}{0}}
\newcommand*{\im}{\includegraphics}
\newcommand*{\lt}{\left}
\newcommand*{\rt}{\right}

\begin{document}

\title{5.1 - 5.4 quiz}
\author{Juan J. Moreno Santos}
\date{January 2024}

\maketitle
\section{Bleh}

1.\begin{align}
    \set
    &\frac{dy}{dx}(\ln(\ln x^{12}))\\
    &=\frac{1}{ln(x^{12})}\frac{d}{dx}(\ln(x^{12}))\\
    &=\frac{1}{ln(x^{12})}\cdot\frac{12}{x}\\
    &=\frac{12}{x\ln(x^{12})}
\end{align}

2.x-coordinate(s) of any relative extrema and inflection points of the function $y=x^5\ln(\frac{x}{9})$.



3.The relationship between the number of decibels $\beta$ and the intensity $I$ of sound in watts per centimeter squared is 
$\beta=10\log_{10}\lt(\frac{I}{10^{-16}}\rt)$ Determine the number of decibels of a sound with an intensity of $10^{-9}$ watts per square centimeter.
\begin{align}
    \set
    &\beta=10\log_{10}\lt(\frac{I}{10^{-16}}\rt)\\
    &=\frac{10}{\ln 10}(\ln I+16\ln 10)\\
    &=160+10\log_{10}I\\
    \beta(10^{-10})&=\frac{10}{\ln 10}(\ln 10^{-9}+16\ln 10)\\
    &=\frac{10}{\ln 10}(-9\ln 10+16\ln 10)\\
    &=\frac{10}{\ln 10}(7\ln 10)\\
    &=70\,\,\text{decibels}
\end{align}

4.\begin{align}
    \set
    &\int\frac{x^2+16x=6}{x^3+24x^2+18x-1}dx,\,\, u=x^3+24x^2+18x-1,\,\,  \frac{du}{3}=\frac{3x^2+48x+18-1dx}{3}=x^2+16+6dx\\
    &=\int\frac{1}{u}\frac{du}{3}\\
    &=\frac{1}{3}\int\frac{1}{u}du
    &=ln|u|+C
\end{align}

5.\begin{align}
    \set
    df
\end{align}

5.\begin{align}
    \set
    df
\end{align}

7.Use the Horizontal Line Test to determine whether the following statement is true or false.
The function is one-to-one on its entire domain and therefore has an inverse function.\\
\indent True.

9.\begin{align}
    \set
    f(x)&=x+2,\,\, g(x)=4x-7\\
    &(g^{-1}\cdot f^{-1})(x)\\
    &=
\end{align}

10.\begin{align}
    \set
    y&=x^5e^{x^9}\\
    \frac{dy}{dx}&=
\end{align}

12.\begin{align}
    \set
    &\int_{1}^{7}\frac{e^{7\sqrt[]{x}}}{\sqrt[]{x}}dx,\,\, u=7\sqrt[]{x},\,\, du=\frac{7}{2\sqrt[]{x}}dx\\
\end{align}

\end{document}