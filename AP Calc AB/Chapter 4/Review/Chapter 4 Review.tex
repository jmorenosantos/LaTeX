\documentclass[11pt]{article}
\usepackage{amsmath, amssymb, amsfonts,  graphicx, enumerate, float, wrapfig, hyperref}
\usepackage[margin=0.5in]{geometry}
\graphicspath{{./}}
\newcommand*{\vs}{\vspace{1cm}}
\newcommand*{\next}{\noindent}
\newcommand*{\set}{\setcounter{equation}{0}}
\newcommand*{\im}{\includegraphics}
\newcommand*{\lt}{\left}
\newcommand*{\rt}{\right}

\begin{document}

\title{Chapter 4 Review}
\author{Juan J. Moreno Santos}
\date{December 2023}

\maketitle

\section{Free Response}
\subsection{4.1 Antiderivatives and Indefinite Integration - p.257}
71. A ball is thrown vertically upward from a height of 6 feet with an
initial velocity of 60 feet per second. How high will the ball go? Use $a(t)=-32$ feet per
second per second as the acceleration due to gravity. 

\begin{align}
    \set
    h(t)&=\frac{1}{2}at^2+v_0t+h_0\\
    \int a(t)&=\int -32\\
    v(t)&=-32t+C\\
    60&=-32(0)+C\\
    v(t)&=-32t+60\\
    h(t)&=\frac{1}{2}(-32)t^2+60t+C\\
    6&=\frac{1}{2}(-32)(0)^2+60(0)+C\\
    \therefore h(t)&=\frac{1}{2}(-32)t^2+(60)t+6
\end{align}
Remember that $x=t=\frac{-b}{2a}$ can be used. With the t-value, we can calculate the max height.
The maximum height is 62.25.

\vs\next
77. A baseball is thrown upward from a height of 2 meters with
an initial velocity of 10 meters per second. Determine its
maximum height. Use $a(t)=-9.8$ meters
per second per second as the acceleration due to gravity.
(Neglect air resistance.)
\begin{align}
    \set
    y&=\frac{1}{2}(-9.8)t^2+10t+2\\
    x&\frac{-b}{2a}=\frac{-10}{2(-4.9)}=1.02
\end{align}
Let $t=1.02$. The maximum height is 7.1m.

\subsection{4.2 Area - p.268}
37. Find the limit of $s(n) as n\to\infty$.
\begin{align}
    \set
    s(n)&=\frac{81}{n^4}\lt(\frac{n^2(n+1)^2}{4}\rt)\\
    \lim_{n\to\infty}\frac{81}{n^4}\lt(\frac{n^2(n+1)^2}{4}\rt)&=\lim_{n\to\infty}\frac{81n^4+162n^3+81n^2}{4n^4}\lt(\frac{\frac{1}{n^4}}{\frac{1}{n^4}}\rt)\\
    &=\lim_{n\to\infty}\frac{81+\frac{162}{n}}+\frac{81}{n^2}\\
    &=\frac{81}{4}
\end{align}

\subsection{4.3 Riemann Sums and Definite Integrals - p.279}
41. Given $\int_{0}^{5}f(x)dx=10$ and $\int_{5}^{7}f(x)dx=3$, evaluate
\begin{enumerate}[(a)]
    \item
        \begin{align}
            \set
            \int_{0}^{7}f(x)dx=13
        \end{align}    
    \item
        \begin{align}
            \set
            \int_{5}^{0}f(x)dx=-10
        \end{align}
    \item
        \begin{align}
            \set
            \int_{5}^{5}f(x)dx=0
        \end{align}
    \item
        \begin{align}
            \set
            \int_{0}^{5}3f(x)dx=30
        \end{align}
\end{enumerate}

42. Given $\int_{0}^{3}f(x)dx=4$ and $\int_{3}^{6}f(x)=-1$ evaluate
\begin{enumerate}[(a)]
    \item
        \begin{align}
            \set
            \int_{0}^{6}f(x)dx=3
        \end{align}
    \item
        \begin{align}
            \set
            \int_{6}^{3}f(x)dx=-1
        \end{align}
    \item
        \begin{align}
            \set
            \int_{3}^{3}f(x)dx=0
        \end{align}
    \item
        \begin{align}
            \set
            \int_{3}^{6}-5f(x)dx=5
        \end{align}
\end{enumerate}

\subsection{4.4 The Fundamental Theorem of Calculus - p.293}
46. Find the value(s) of $c$ guaranteed by the
Mean Value Theorem for Integrals for the function over the
given interval.
\begin{align}
    \set
    f(x)&=\frac{9}{x^3},\,\,\,[1, 3]\\
\end{align}
Remember that the average height of a function on an interval is given by
\begin{align}
    \frac{1}{b-a}\int_{a}^{b}f(x)dx=f(c)\\
\end{align} 
Therefore,
\begin{align}
    &\frac{1}{2}\int_{1}^{3}9x^{-3}dx\\
    &=\frac{9}{2}\cdot\frac{x^{-2}}|_1^3\\
    &=-\frac{9}{4}(3)^{-2}-\lt(-\frac{9}{4}(1)^{-2}\rt)\\
    &=\frac{1}{4}+\frac{9}{4}=\frac{8}{4}=2
\end{align}
Finding c:
\begin{align}
    2&=\frac{9}{c^3}\\
    2c^3&=9\\
    c^3&=4.5\\
    c&=\sqrt[3]{4.5}
\end{align}

\subsection{4.5 Integration by Substitution - p.306}
41. Solve the differential equation\\
1. Start by separating the integral
\begin{align}
    \set
    \frac{dy}{dx}&=\frac{x+1}{(x^2+2x-3)^2}\\
    dy&=\frac{x+1}{(x^2+2x-3)^2}dx\\
\end{align}

2. Integrate
\begin{align}
    \set
    \int dy&=\int \frac{x+1}{(x^2+2x-3)^2}dx,\,\,\, u=x^2+2x-3,\,\,\, du=2x+2dx\therefore \frac{1}{2}du=x+1dx\\
    y&=\frac{1}{2}\int\lt(\frac{1}{u^2}\rt)du\\
    &=\int u^{-2}du\\
    &=-\frac{1}{2}u^{-1}+C\\
    &=-\frac{1}{2(x^2+2x-3)}+C
\end{align}

\subsection{4.6 Numerical Integration - p. 316}
8. Use the Trapezoidal Rule to approximate the value of the definite integral for the given
value of $n$ Round your answer to four decimal places and
compare the results with the exact value of the definite integral.
\begin{align}
    \set
    &\int_{1}^{4}(4-x^2)dx,\,\,\, n=6\\
    &=\frac{4-1}{2(6)}(f(1)+2f(\frac{3}{2})+2f(2)+2\frac{5}{2}+2f(3)+2f(\frac{7}{2})+f(4))\\
    &=-9
\end{align}

\end{document}