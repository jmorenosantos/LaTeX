\documentclass[11pt]{article}
\usepackage{amsmath, amssymb, amsfonts,  graphicx, enumerate, float, wrapfig, hyperref}
\usepackage[margin=0.5in]{geometry}
\graphicspath{{./}}
\newcommand*{\vs}{\vspace{1cm}}
\newcommand*{\next}{\noindent}
\newcommand*{\set}{\setcounter{equation}{0}}
\newcommand*{\im}{\includegraphics}

\begin{document}

\title{4.1 Antiderivatives and Indefinite Integration}
\author{Juan J. Moreno Santos}
\date{November 2023}

\maketitle

\section{Verify that the statement by showing that the derivative of the right side equals the integrand of the left side.}
2.\begin{align}
    \set
    \int \left(8x^3+\frac{1}{2x^2}\right)dx&=2x^4-\frac{1}{2}x^{-1}+C\\
    \frac{d}{dx}\left(2x^4-\frac{1}{2x}+C\right)&=\frac{d}{dx}\left(2x^4-\frac{1}{2}x^{-1}+C\right)\\
    &=8x^3+\frac{1}{2}x^{-2}\\
    &=8x^3+\frac{1}{2x^2}
\end{align}

\section{Find the general solution of the differential equation and check the result by differentiation.}
6.\begin{align}
    \set
    \frac{dr}{d\theta}&=\pi\\
    r&=\pi\theta+C\because\frac{d}{d\theta}[\pi\theta+C]=\pi\\
\end{align}

\vs\next
8.\begin{align}
    \set
    \frac{dy}{dx}&=2x^{-3}\\
    y&=\frac{2x^{-2}}{-2}+C\\
    &=\frac{-1}{x^2}+C\because \frac{d}{dx}\left(\frac{-1}{x^2}+C\right)=2x^{-3}
\end{align}

\section{Complete the table.}
\begin{flushleft}
    \begin{table}[h]
        \begin{tabular}{|l|l|l|l|}
        \hline
        Original integral & Rewrite & Integrate & Simplify\\\hline
        10. $\int\frac{1}{x\sqrt[]{x}}dx$ & $\frac{1}{4}\int x^{-2}dx$ & $\frac{1}{4}\frac{x^{-1}}{-1}+C$ & $-\frac{1}{4x}+C$\\\hline
        14. $\int \frac{1}{(3x)^2}dx$ & $\frac{1}{9}\int x^{-2}dx$ & $\frac{1}{9}\left(\frac{x^{-1}}{-1}\right)+C$ & $\frac{-1}{9x}+C$\\\hline
        \end{tabular}
    \end{table}
\end{flushleft}

\section{Find the indefinite integral and check the result by differentiation.}
16.\begin{align}
    \set
    &\int(13-x)dx\\
    &=13x-\frac{x^2}{2}+C\because\frac{d}{dx}\left(13x-\frac{x^2}{2}+C\right)=13-x
\end{align}

\vs\next
20.\begin{align}
    \set
    &\int(x^3-10x-3)dx\\
    &=\frac{x^4}{4}-5x^2-3x+C\because\frac{d}{dx}\left(\frac{x^4}{4}-5x^2-3x+C\right)=x^3-10x-3
\end{align}

\vs\next
24.\begin{align}
    \set
    &\int(\sqrt[4]{x^3}+1)dx\\
    &=\int(x^{3/4}+1)dx\\
    &=\frac{4}{7}x^{7/4}+x+C\because\frac{d}{dx}\left(\frac{4}{7}x^{7/4}+x+C\right)=x^{3/4}+1=\sqrt[4]{x^3}+1
\end{align}

\vs\next
28.\begin{align}
    \set
    &\int\frac{x^2+2x-3}{x^4}dx\\
    &=\int(x^{-2}+2x^{-3}-3x^{-4})dx\\
    &=\frac{x^{-1}}{-1}+\frac{2x^{-2}}{-2}-\frac{3x^{-3}}{-3}+C\\
    &=\frac{-1}{x}-\frac{1}{x^2}+\frac{1}{x^3}+C\because\frac{d}{dx}\left(\frac-{1}{x}-\frac{1}{x^2}+\frac{1}{x^3}+C\right)=x^{-2}+2x^{-3}-3x^{-4}=\frac{x^2+2x-3}{x^4}
\end{align}

\vs\next
32.\begin{align}
    \set
    &\int(1+3t)t^2dt\\
    &=\int(t^2+3t^3)dt\\
    &=\frac{1}{3}t^3+\frac{3}{4}t^4+C\because\frac{d}{dt}\left(\frac{1}{3}t^3+\frac{3}{4}t^4+C\right)=t^2+3t^3=(1+3t)t^2
\end{align}

\vs\next
36.\begin{align}
    \set
    &\int(t^2-\cos t)dt\\
    &=\frac{t^3}{3}-\sin t+C\because\frac{d}{dt}\left(\frac{t^3}{3}-\sin t+C\right)=t^2-\cos t
\end{align}

\vs\next
40.\begin{align}
    \set
    &\int\sec y(\tan y-\sec y)dy\\
    &=\int(\sec\tan y-\sec^2y)dy\\
    &=\sec y-\tan y+C\because\frac{d}{dy}(\sec y-\tan y+C)=\sec y\tan y-\sec^2 y=\sec y(\tan y-\sec y)
\end{align}

\vs\next
44.\begin{align}
    \set
    &\int\frac{\sin x}{1-\sin^2 x}dx\\
    &=\int\frac{\sin x}{\cos^2 x}dx\\
    &=\int\tan x\sec xdx\\
    &=\sec x+C
\end{align}

\section{The graph of the derivative of a function is given. Sketch the graphs of two functions that have the given derivative.}
46.\\\im{46.png}
\begin{align}
    \set
    f'(x)&=x\\
    f(x)&=\frac{x^2}{2}+C\therefore f(x)=\frac{x^2}{2}+1\,\, \frac{x^2}{2}
\end{align}
\im{46b.png}

\section{Find the equation of $y$. given the derivative and the indicated point on the curve.}
50.\[\frac{dy}{dx}=2(x-1)\]
\im{50.png}
\begin{align}
    &=2x-2,\,\,(3, 2)\\
    y&=\int2(x-1)dx\\
    &=x^2-2x+C\\
    2&=(3)^2-2(3)+C\Rightarrow C=-1\\
    y&=x^2-2x-1
\end{align}

\section{A differential equation, a
point, and a slope field are given. A slope field (or direction field)
consists of line segments with slopes given by the differential
equation. These line segments give a visual perspective of the
slopes of the solutions of the differential equation. (a) Sketch
two approximate solutions of the differential equation on the
slope field, one of which passes through the indicated point. (b) Use integration to find the particular
solution of the differential equation and use a graphing utility to
graph the solution. Compare the result with the sketches in
part (a).}

52.\[\frac{dy}{dx}=x^2-1,\,\,(-1, 3)\]
\im{52.png}
\begin{enumerate}[(a)]
    \item\im{52b.png}
    \item\begin{align}
        \set
        \frac{dy}{dx}&=x^2-1,\,\,(-1, 3)\\
        y&=\frac{x^3}{3}-x+C\\
        3&=\frac{(-1)^3}{3}-(-1)+C\\
        C&=\frac{7}{3}\\
        y&=\frac{x^3}{3}-x+\frac{7}{3}
    \end{align}
\end{enumerate}

54.\[\frac{dy}{dx}=-\frac{1}{x^2},\,\, x>0,\,\, (1, 3)\]
\im{43b.png}
\begin{enumerate}[(a)]
    \item\im{54.png}
    \item \begin{align}
        \set
        y&=\int-\frac{1}{x^2}dx\\
        &=\int-x^{-2}dx\\
        &=\frac{-x^{-1}}{-1}+C\\
        &=\frac{1}{x}+C\\
        3&=\frac{1}{1}+C\Rightarrow C=2\\
        y&=\frac{1}{x}+2
    \end{align}
\end{enumerate}

\section{(a) Use a graphing utility
to graph a slope field for the differential equation, (b) use
integration and the given point to find the particular solution of
the differential equation, and (c) graph the solution and the
slope field in the same viewing window.}
56.\begin{align}
    \set
    \frac{dy}{dx}&=2\sqrt[]{x},\,\,(4, 12)\\
    y&=\int 2x^{1/2}dx\\
    &=\frac{4}{3}x^{3/2}+C\\
    12&=\frac{4}{3}(4)^{3/2}+C\\
    &=\frac{4}{3}(8)+C\\
    &=\frac{32}{3}+C\Rightarrow C=\frac{4}{3}\\
    y&=\frac{4}{3}x^{3/2}+\frac{4}{3}
\end{align}

\section{Solve the differential equation.}
58.\begin{align}
    \set
    g'(x)&=6x^2,\,\, g(0)=-1\\
    g(x)&=\int 6x^2dx=2x^3+C\\
    g(0)&=-1=2(0)^3+C\Rightarrow C=-1\\
    g(x)&=2x^3-1
\end{align}

\vs\next
62.\begin{align}
    \set
    f''(x)&=x^2,\,\, f'(0)=8,\,\, f(0)=4\\
    f'(x)&=\int x^2dx=\frac{1}{3}x^3+C_1\\
    f'(0)&=0+C_1=8\Rightarrow C_1=8\\
    f'(x)&=\frac{1}{3}x^3+8\\
    f(x)&=\int\left(\frac{1}{3}x^3+8\right)dx=\frac{1}{12}x^4+8x+C_2\\
    f(0)&=0+0+C_2=4\Rightarrow C_2=4\\
    f(x)&\frac{1}{12}x^4+8x+4
\end{align}

\section{Capstone}
70. Use the graph of $f'$ shown in the figure to answer the following, given that $f(0)=-4$.\\
\im{70.png}\\
\begin{enumerate}[(a)]
    \item Approximate the slope of $f$ at $x=4$.
    \begin{align}
        \set
        f'(4)\approx -1
    \end{align}
    \item Is it possible that $f(2)=-1$? Explain.\\
    No because the tangent lines' slopes on [0, 2] are greater than 2, and $f$ would have to increase more than 4 on [0, 4].
    \item Is $f(5)-f(4)>0$? Explain.
    No because $f$ decreases on [4, 5].
    \item Approximate the value of $x$ where $f$ is maximum. Explain.\\
    $f$ is maximum at $x=3.5$ because $f'(3.5)=0$.
    \item Approximate any intervals in which the graph of is
    concave upward and any intervals in which it is concave
    downward. Approximate the coordinates of any
    points of inflection.\\
    $f$ is concave upward when $f'$ increases on $(-\infty, 1$ and $(5, \infty)$. $f$ is concave downward on (1, 5). There are points of inflection at $x=1, 5$. 
    \item Approximate the coordinate of the minimum of $f''(x)$.\\
    $f''(x)$ is minimum at $x=3$.
    \item Sketch an approximate graph of $f$.\\
    \im{70b.png}
\end{enumerate}

\section{Vertical Motion. Use $a(t)=-32$ feet per
second per second as the acceleration due to gravity. (Neglect
air resistance.)}
74. A balloon, rising vertically with a velocity of 16 feet per
second, releases a sandbag at the instant it is 64 feet above the
ground.
\begin{enumerate}[(a)]
    \item How many seconds after its release will the bag strike the
    ground?
        \begin{align}
            \set
            v_0&=16\text{ft}/\text{sec}\\
            s_0&=64\text{ft}\\
            s(t)&=-16t^2+16t+64=0\\
            -16(t^2-t-4)&=0\\
            t&=\frac{1+\sqrt[]{17}}{2}\approx 2.562\text{seconds}\\
        \end{align}
    \item At what velocitiy will it hit the ground?
        \begin{align}
            \set
            v(t)&=s'(t)=-32t+16\\
            v\left(\frac{1+\sqrt[]{17}}{2}\right)&=-32\left(\frac{1+\sqrt[]{17}}{2}\right)+16\\
            &=-16\sqrt[]{17}\approx -65.970\text{ft/sec}
        \end{align}
\end{enumerate}

\section{Vertical Motion. Use $a(t)=-9.8$ meters
per second per second as the acceleration due to gravity.
(Neglect air resistance.)}

76. The Grand Canyon is 1800 meters deep at its deepest point. A
rock is dropped from the rim above this point. Write the height
of the rock as a function of the time $t$ in seconds. How long will
it take the rock to hit the canyon floor?
\begin{align}
    \set
    f(t)&=0=-4.9t^2+1800\\
    4.9t^2&=1800\\
    t^2&=\frac{1800}{4.9}\Rightarrow t\approx 9.2\text{seconds}
\end{align}

\section{Rectilinear Motion. Consider a particle
moving along the $x$-axis where $x(t)$ is the position of the particle
at time $t$, $x'(t)$ is its velocity, and $x''(t)$  is its acceleration.}

82.\begin{enumerate}[(a)]
    \item Find the velocity and acceleration of the particle.
        \begin{align}
            x(t)&=(t-1)(t-3)^2,\,\, 0\leq t\leq 5\\
            &=t^3-7t^2+15-9\\
            v(t)&=x'(t)=3t^2-14t+15=(3t-5)(t-3)\\
            a(t)&=v'(t)=6t-14
        \end{align}
    \item Find the open intervals on which the particle is moving to
    the right.
        \begin{align}
            v(y)>0\,\,\text{when}0<t<\frac{5}{3}\,\,\text{and}\,\, 3<t<5\\
        \end{align}
    \item Find the velocity of the particle when the acceleration is 0.
        \begin{align}
            a(t)&=6t-14=0\,\,\text{when}\,\, t=\frac{7}{3}\\
            v(\frac{7}{3})&=\left(3\left(\frac{7}{3}\right)-5\right)\left(\frac{7}{3}-3\right)=2\left(-\frac{2}{3}\right)=-\frac{4}{3}
        \end{align}
\end{enumerate}

\section{Rectilinear Motion. Consider a particle
moving along the $x$-axis where $x(t)$ is the position of the particle
at time $t$, $x'(t)$ is its velocity, and $x''(t)$ is its acceleration.}
84. A particle, initially at rest, moves along the $x$-axis such that its acceleration at time $t>0$ is given by $a(t)=\cos t$. At the time $t=0$, its position $x=3$
\begin{enumerate}[(a)]
    \item Find the velocity and position functions for the particle.
        \begin{align}
            \set
            a(t)&=\cos t\\
            v(t)&=\int a(t)dt\\
            &=\int\cos tdt\\
            &=\sin t+C_1=\sin t\\
            f(t)&=\int v(t)dt=\int\sin tdt=-\cos +C_2\\
            f(0)&=3=-\cos(0)+C_2=-1+C_2\Rightarrow C_2=4\\
            f(t)&=-\cos t+4
        \end{align}
    \item Find the values of $t$ for which the particle is at rest.
        \begin{align}
            \set
            v(t)=0=\sin t,\,\, t=k\pi,\,\, k=0, 1, 2,...
        \end{align}
\end{enumerate}

\section{Deceleration}
86. A car traveling at 45 miles per hour is brought
to a stop, at constant deceleration, 132 feet from where the
brakes are applied.
\begin{enumerate}[(a)]
    \item How far has the car moved when its speed has been reduced
    to 30 miles per hour?
        \begin{align}
            \set
            v(0)&=45\text{mi/h}=66\text{ft/sec}\\
            &30\text{mi/h}=44\text{ft/sec}\\
            &15\text{mi/h}=22\text{ft/sec}\\
            a(t)&=-a\\
            v(t)&=-at+66\\
            s(t)&=-\frac{a}{2}t^2+66t\\
            v(t)&=0\,\,\text{after 132 ft}\\
            -at+66&=0\,\,\text{when}\,\, t=\frac{66}{a}\\
            s\left(\frac{66}{a}\right)&=-\frac{a}{2}\left(\frac{66}{a}\right)^2+66\left(\frac{66}{a}\right)\\
            &=132\,\,\text{when}\,\, a=\frac{33}{2}=16.5\\
            a(t)&=-16.5\\
            v(t)&=-16.5t+66\\
            s(t)&=-8.25t^2+66t
            -16.5t+66&=44\\
            t&=\frac{22}{16.5}\approx 1.333\\
            s\left(\frac{22}{16.5}\right)&\approx 73.33
        \end{align}
    \item How far has the car moved when its speed has been reduced
    to 15 miles per hour?
        \begin{align}
            \set
            -16.5t+66&=22\\
            t&=\frac{44}{16.5}\approx 2.667\\
            s\left(\frac{44}{16.5}\right)&\approx 117.33
        \end{align}
    \item Draw the real number line from 0 to 132, and plot the points
    found in parts (a) and (b). What can you conclude?
        \im{86.png}\\
        Less distance is needed to reach the next reduction of speed every time.
\end{enumerate}

\section{Determine whether the
statement is true or false. If it is false, explain why or give an
example that shows it is false.}

90. Each antiderivative of an $n$th-degree polynomial function is an
$(n+1)$th-degree polynomial function.
\indent True.

\vs\next
91. If $p(x)$ is a polynomial function, then $p$ has exactly one
antiderivative whose graph contains the origin.\\
\indent True.

\vs\next
92. If $F(x)$ and $G(x)$ are antiderivatives of $f(x)$, then $F(x)=G(x)+C$.\\
\indent True.

\vs\next
93. If $f'(x)=g(x)$, then $\int g(x)dx=f(x)_C$.\\
\indent True.

\vs\next
94. $\int f(x)g(x)dx=\int f(x)dx\int g(x)dx$\\
\indent False.
\[\int x\cdot xdx\neq\int xdx\cdot\int xdx\because \frac{x^3}{3}+C\neq\left(\frac{x^2}{2}+C_1\right)\left(\frac{x^2}{2}+C_2\right)\]

\vs\next
95. The antiderivative of $f(x)$ is unique.\\
\indent False. It has an infinite number of integrals, each being different by a constant.





























\end{document}