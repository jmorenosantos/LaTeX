\documentclass[11pt]{article}
\usepackage{amsmath, amssymb, amsfonts,  graphicx, enumerate, float, wrapfig, hyperref}
\usepackage[margin=0.5in]{geometry}
\graphicspath{{./}}
\newcommand*{\vs}{\vspace{1cm}}
\newcommand*{\next}{\noindent}
\newcommand*{\set}{\setcounter{equation}{0}}
\newcommand*{\im}{\includegraphics}

\begin{document}

\title{4.2 Area}
\author{Juan J. Moreno Santos}
\date{December 2023}

\maketitle

\section{Find the sum. Use the summation capabilities of a graphing utility to verify your result.}
2.\begin{align}
    &\sum_{k=5}^{8}k(k-4)\\
    &=5(1)+6(2)+7(3)+8(4)=70
\end{align}

\vs\next
6.\begin{align}
    \set
    &\sum_{i=1}^{4}((i-1)^2+(i+1)^3)\\
    &=(0+8)+(1+27)+(4+64)+(9+125)\\
    &=238\\
\end{align}

\section{Use sigma notation to write the sum.}
8.\begin{align}
    \set
    &\frac{9}{1+1}+\frac{9}{1+2}+\frac{9}{1+3}\cdots+\frac{9}{1+14}\\
    &=\sum_{i=1}^{14}\frac{9}{1+i}
\end{align}

12.\begin{align}
    \set
    &\left(1-\left(\frac{2}{n}-1\right)^2\right)\left(\frac{2}{n}\right)+\cdots+\left(1-\left(\frac{2n}{n}-1\right)^2\right)\left(\frac{2}{n}\right)\\
    &=\frac{2}{n}\sum_{i=1}^{n}\left(1-\left(\frac{2i}{n}-1\right)^2\right)
\end{align}

\section{Use the properties of summation and
Theorem 4.2 (summation formulas) to evaluate the sum. Use the summation capabili-
ties of a graphing utility to verify your result.}

16.\begin{align}
    \set
    &\sum_{i=1}^{30}-18\\
    &=(-18)(30)\\
    &=-540
\end{align}

\vs\next
20.\begin{align}
    \set
    &\sum_{i=1}^{10}(i^2-1)\\
    &=\sum_{i=1}^{10}i^2-\sum_{i=1}^{10}1\\
    &=\left(\frac{10(11)(21)}{6}\right)-10\\
    &=375
\end{align}

\vs\next
26. Consider the function $g(x)=x^2+x-4$.
\begin{enumerate}[(a)]
    \item Estimate the are between the graph of $g$ and the $x$-axis between $x=2$ and $x=4$ using four rectangles and right endpoints. Sketch the graph and the rectangles.\\
    \im{26.png}\\
    The $\Delta x$ width of each rectangle is $\frac{1}{2}$. The right endpoints yield the heights.
        \begin{align}
            \set
            \text{Area}&\approx\frac{1}{2}\left(\left(\left(\frac{5}{2}\right)^2+\left(\frac{5}{2}\right)-4\right)+(3^2+3-4)+\left(\left(\frac{7}{2}\right)^2+\frac{7}{2}-4\right)+(4^2+4-4)\right)\\
            &=\frac{81}{4}\\
            &=20.25
        \end{align}
\end{enumerate}

\section{Use left and right endpoints and the given
number of rectangles to find two approximations of the area of
the region between the graph of the function and the -axis over
the given interval.}
28.$f(x)=9-x,\,\,[2, 4],\,\,\text{6 rectangles}$
\begin{align}
    \set
    \Delta x&=\frac{4-2}{6}=\frac{1}{3}\\
    \text{Left endpoints: Area}&\approx\frac{1}{3}\left(7+\frac{20}{3}+\frac{19}{3}+6+\frac{17}{3}+\frac{16}{3}\right)=\frac{37}{3}\\
    \text{Left endpoints: Area}&\approx\frac{1}{3}\left(\frac{20}{3}+\frac{19}{3}+6+\frac{17}{3}+\frac{16}{3}+\frac{15}{3}\right)=\frac{35}{3}\\
    &\frac{35}{3}<\text{Area}<\frac{37}{3}
\end{align}

\vs\next
30.$g(x)=x^2+1,\,\,[1, 3],\,\,\text{8 rectangles}$
\begin{align}
    \set
    \Delta x&=\frac{3-1}{8}=\frac{1}{4}\\
    \text{Left endpoints: Area}&\approx\frac{1}{4}\left(2+\frac{41}{16}+\frac{13}{4}+\frac{65}{16}+5+\frac{97}{16}+\frac{29}{4}+\frac{137}{16}\right)=\frac{155}{16}=9.6875\\
    \text{Left endpoints: Area}&\approx\frac{1}{4}\left(\frac{41}{16}+\frac{13}{4}+\frac{65}{16}+5+\frac{97}{16}+\frac{29}{4}+\frac{137}{16}+10\right)=11.6875\\
    &9.6875<\text{Area}<11.6875
\end{align}

\vs\next
32.$g(x)=\sin x,\,\,[0, \pi],\,\,\text{4 rectangles}$
\begin{align}
    \set
    \Delta x&=\frac{\pi-0}{6}=\frac{\pi}{6}\\
    \text{Left endpoints: Area}&\approx\frac{\pi}{6}\left(\sin 0+\sin\frac{\pi}{6}+\sin\frac{\pi}{3}+\sin\frac{\pi}{2}+\sin\frac{2\pi}{3}+\sin\frac{5\pi}{6}\right)\approx 1.9541\\
    \text{Left endpoints: Area}&\approx\frac{\pi}{6}\left(\sin\frac{\pi}{6}+\sin\frac{\pi}{3}+\sin\frac{\pi}{2}+\sin\frac{2\pi}{3}+\sin\frac{5\pi}{6}+\sin\pi\right)\approx 1.9541\\
    &\text{The answers are the same by symmetry. The exact area of 2 is larger.}
\end{align}

\section{Bound the area of the shaded region by
approximating the upper and lower sums. Use rectangles of
width 1.}
36.\\\im{36.png}\\
\begin{align}
    \set
    S&=\left(5+2+1+\frac{2}{3}+\frac{1}{2}\right)(1)=\frac{55}{6}\\
    s&=\left(2+1+\frac{2}{3}+\frac{1}{2}+\frac{1}{3}\right)(1)=\frac{9}{2}
\end{align}

\section{Use upper and lower sums to approximate
the area of the region using the given number of subintervals (of
equal width).}
44.\\\im{44.png}\\
\begin{align}
    \set
    S(5)&=1\left(\frac{1}{5}\right)+\sqrt[]{1-\left(\frac{1}{5}\right)^2}\left(\frac{1}{5}\right)+\sqrt[]{1-\left(\frac{2}{5}\right)^2}\left(\frac{1}{5}\right)+\sqrt[]{1-\left(\frac{3}{5}\right)^2}\left(\frac{1}{5}\right)+\sqrt[]{1-\left(\frac{4}{5}\right)^2}\left(\frac{1}{5}\right)\\
    &=\frac{1}{5}\left(1+\frac{\sqrt[]{24}}{5}+\frac{\sqrt[]{21}}{5}+\frac{\sqrt[]{16}}{5}+\frac{\sqrt[]{9}}{5}\right)\approx 0.859\\
    s(5)&=\sqrt[]{1-\left(\frac{1}{5}\right)^2}\left(\frac{1}{5}\right)+\sqrt[]{1-\left(\frac{2}{5}\right)^2}\left(\frac{1}{5}\right)+\sqrt[]{1-\left(\frac{3}{5}\right)^2}\left(\frac{1}{5}\right)+\sqrt[]{1-\left(\frac{4}{5}\right)^2}\left(\frac{1}{5}\right)+0\approx 0.659
\end{align}

\section{Use the Midpoint Rule $\text{Area}\approx\sum_{i=1}^{n}f\left(\frac{x_i+x_{i-1}}{2}\right)\Delta x$ with $n=4$ to approximate the area of the region bounded
by the graph of the function and the $x$-axis over the given
interval.}
74.\begin{align}
    \set
    f(x)&=x^2+4x,\,\,[0, 4],\,\, n=4\\
    \text{Let}\,\, c_i&=\frac{x_i+x_{i-1}}{2}\\
    \Delta x&=1,\,\, c_1=\frac{1}{2},\,\, c_2=\frac{3}{2},\,\, c_3=\frac{5}{2},\,\, c_4=\frac{7}{2}\\
    \text{Area}&\approx \sum_{i=1}^{n}f(c_i)\Delta x=\sum_{i-1}^{4}(c_i^2+4c_i)(1)\\
    &=\left(\left(\frac{1}{4}+2\right)+\left(\frac{9}{4}+6\right)+\left(\frac{25}{4}+10\right)+\left(\frac{49}{4}+14\right)\right)\\
    &=53
\end{align}

\vs\next
76.\begin{align}
    \set
    f(x)&=\sin x,\,\, 0\leq x\leq\frac{\pi}{2},\,\, n=4\\
    \text{Let}\,\, c_i&=\frac{x_i+x_{i-1}}{2}\\
    \Delta x&=\frac{\pi}{8},\,\, c_1=\frac{\pi}{16},\,\, c_2=\frac{3\pi}{16},\,\, c_3=\frac{5\pi}{16},\,\, c_4=\frac{7\pi}{16}\\
    \text{Area}&\approx\sum_{i=1}^{n}f(c_i)\Delta x\\
    &=\sum_{i=1}^{4}(\sin c_i)\left(\frac{\pi}{8}\right)=\frac{\pi}{8}\left(\sin\frac{\pi}{16}+\sin\frac{3\pi}{16}+\sin\frac{5\pi}{16}+\sin\frac{7\pi}{16}\right)\approx 1.006
\end{align}

\section{Capstone}
86. Consider a function $f(x)$ that is increasing on the interval $[1, 4]$
The interval $[1, 4]$ is divided into 12 subintervals.\\
\begin{enumerate}[(a)]
    \item What are the left endpoints of the first and last
    subintervals?
        \begin{align}
            \set
            \Delta x=\frac{4-1}{12}=\frac{1}{4}
        \end{align}
        \indent The left endpoint of the first subinterval is 1, and the left endpoint of the last subinterval is $4-\frac{1}{4}=\frac{15}{4}$.

    \item What are the right endpoints of the first two subintervals?\\
        \indent The right end points of the first two subintervals are $1+\frac{1}{4}=\frac{5}{4}$ and $1+2\left(\frac{1}{4}\right)=\frac{3}{2}$.
    \item When using the right endpoints, will the rectangles lie above or below the graph of $f(x)$> Use a graph to explain your answer.\\
        \indent When using right endpoints, the rectangles will be above the curve.
    \item What can you conclude about the heights of the rectangles if a function is constant on the given interval?\\
        \indent The height of the rectangles are the same for a constant function.
\end{enumerate}

































\end{document}