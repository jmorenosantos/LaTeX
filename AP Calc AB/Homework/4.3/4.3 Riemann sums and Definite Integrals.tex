\documentclass[11pt]{article}
\usepackage{amsmath, amssymb, amsfonts,  graphicx, enumerate, float, wrapfig, hyperref}
\usepackage[margin=0.5in]{geometry}
\graphicspath{{./}}
\newcommand*{\vs}{\vspace{1cm}}
\newcommand*{\next}{\noindent}
\newcommand*{\set}{\setcounter{equation}{0}}
\newcommand*{\im}{\includegraphics}

\begin{document}

\title{4.3 Riemann Sums and Definite Integrals}
\author{Juan J. Moreno Santos}
\date{December 2023}

\maketitle

\section{Set up a definite integral that yields the area
of the region. (Do not evaluate the integral.)}
14. $f(x)=6-3x$\\
\im{14.png}\\
\begin{align}
    \set
    \int_{0}^{2}(6-3x)dx
\end{align}

\vs\next
16.$f(x)=x^2$\\
\im{16.png}\\
\begin{align}
    \set
    \int_{0}^{2}x^2dx
\end{align}

\vs\next
18.$f(x)=\frac{4}{x^2+2}$\\
\im{18.png}\\
\begin{align}
    \set
    \int_{-1}^{1}\frac{4}{x^2+2}dx
\end{align}

\vs\next
20.$f(x)=\tan x$\\
\im{20.png}\\
\begin{align}
    \set
    \int_{0}^{\pi/4}\tan xdx
\end{align}

\vs\next
22.$f(y)=(y-2)^2$\\
\im{22.png}\\
\begin{align}
    \set
    \int_{0}^{2}(y-2)^2dy
\end{align}

\section{Sketch the region whose area is given by the
definite integral. Then use a geometric formula to evaluate the
integral $(a>0,$ $r>0)$}
24.\[\int_{-a}^{a}4dx\]
\begin{align}
    \set
    \text{Rectangle}
    A&=bh=2(4)(a)=8a\\
    A&=\int_{-a}^{a}4dx=8a
\end{align}
\im{24.png}

\vs\next
28.\[\int_{0}^{6}(6-x)dx\]
\begin{align}
    \set
    \text{Triangle}
    A&=\frac{1}{2}bh=\frac{1}{2}(6)(6)=18\\
    A&=\int_{0}^{8}(6-x)dx=18
\end{align}
\im{28.png}

\vs\next
30.\[\int_{-a}^{a}(a-|x|)dx\]
\begin{align}
    \set
    \text{Triangle}
    A&=\frac{1}{2}bh=\frac{1}{2}(2a)a=a^2\\
    A&=\int_{-a}^{a}(a-|x|)dx=a^2
\end{align}
\im{30.png}

\vs\next
32.\[\int_{-r}^{r}\sqrt[]{r^2-x^2}dx\]
\begin{align}
    \set
    A&=\frac{1}{2}\pi r^2\\
    A&=\int_{-r}^{r}\sqrt[root]{r^2-x^2}dx=\frac{1}{2}\pi r^2
\end{align}
\im{32.png}

\section{Evaluate the integral using the following values.}
\[\int_{2}^{4}x^3dx=60,\,\,\int_{2}^{4}xdx=6,\,\,\int_{2}^{4}dx=2\]
36.\begin{align}
    \set
    &\int_{2}^{4}25dx\\
    &=25\int_{2}^{4}dx\\
    &=25(2)=50
\end{align}

\vs\next
40.\begin{align}
    \set
    &\int_{2}^{4}(10+4x-3x^3)dx\\
    &=10\int_{2}^{4}dx+4\int_{2}^{4}dx-3\int_{2}^{4}dx\\
    &=10(2)+4(6)-3(60)\\
    &=-136
\end{align}

\section{Given $\int_{0}^{3}f(x)dx=4$ and $\int_{3}^{6}f(x)dx=-1$, evaluate}
42.\begin{enumerate}[(a)]
    \item
        \begin{align}
            \set
            &\int_{0}^{6}f(x)dx\\
            &=\int_{0}^{3}f(x)dx+\int_{3}^{6}f(x)dx\\
            &=4+(-1)=3
        \end{align}
    \item
        \begin{align}
            \set
            &\int_{3}^{6}f(x)dx\\
            &=-\int_{3}^{6}f(x)dx\\
            &=-(-1)=1
        \end{align}
    \item
        \begin{align}
            \set
            &\int_{3}^{3}f(x)dx\\
            &=0
        \end{align}
    \item
        \begin{align}
            \set
            &\int_{3}^{6}-5f(x)dx\\
            &=-5\int_{3}^{6}f(x)dx\\
            &=-5(-1)=5
        \end{align}
\end{enumerate}

\section{Given $\int_{-1}^{1}f(x)dx=0$ and $\int_{0}^{1}f(x)dx=5$, evaluate}
42.\begin{enumerate}[(a)]
    \item
        \begin{align}
            \set
            &\int_{-1}^{0}f(x)dx\\
            &=\int_{-1}^{1}f(x)dx-\int_{0}^{1}f(x)dx\\
            &=0-5=-5
        \end{align}
    \item
        \begin{align}
            \set
            &\int_{0}^{1}f(x)dx-\int_{-1}^{0}f(x)dx\\
            &=5-(-5)=10
        \end{align}
    \item
        \begin{align}
            \set
            &\int_{-1}^{1}3f(x)dx\\
            &=3\int_{-1}^{1}f(x)dx\\
            &=3(0)=0
        \end{align}
    \item
        \begin{align}
            \set
            &\int_{0}^{1}3f(x)dx\\
            &=3\int_{0}^{1}f(x)dx\\
            &=3(5)=15
        \end{align}
\end{enumerate}

\vs\next
46. Use the table of values to estimate $\int_{0}^{6}f(x)dx$. Use three equal
subintervals and the (a) left endpoints, (b) right endpoints, and
(c) midpoints. If $f$ is an increasing function, how does each esti-
mate compare with the actual value? Explain your reasoning.\\
\im{46.png}\\
\begin{enumerate}[(a)]
    \item Left endpoint estimate: $(-6+8+30)(2)=64$
    \item Right endpoint estimate: $(8+30+80)(2)=236$
    \item Midpoint estimate: $(0+18+50)(2)=136$
\end{enumerate}

\section{Think About It}
48. The graph of $f$ consists of line segments, as
shown in the figure. Evaluate each definite integral by using
geometric formulas.\\
\im{48.png}
\begin{enumerate}[(a)]
    \item
        \begin{align}
            \set
            \int_{0}^{1}-f(x)dx=-\int_{0}^{1}f(x)dx=\frac{1}{2}
        \end{align}
    \item
        \begin{align}
            \set
            \int_{3}^{4}3f(x)dx=3(2)=6
        \end{align}
    \item
        \begin{align}
            \set
            \int_{0}^{7}f(x)dx=-\frac{1}{2}+\frac{1}{2}(2)(2)+2+\frac{1}{2}(2)(2)-\frac{1}{2}=5
        \end{align}
    \item
        \begin{align}
            \set
            \int_{5}^{11}f(x)dx=-\frac{1}{2}+2+2+2-4+\frac{1}{2}=2
        \end{align}
    \item
        \begin{align}
            \set
            \int_{4}^{10}f(x)dx=2-4=-2
        \end{align}
\end{enumerate}

\section{Capstone}
52. Find possible values of $a$ and $b$ that make the statement
true. If possible, use a graph to support your answer. (There
may be more than one correct answer.)
\begin{enumerate}[(a)]
    \item
        \begin{align}
            \set
            &\int_{-2}^{1}f(x)dx+\int_{1}^{5}f(x)dx=\int_{-2}^{5}f(x)dx\\
            a&=-2,\,\, b=5
        \end{align}
    \item
        \begin{align}
            \set
            &\int_{-3}^{3}f(x)dx+\int_{3}^{6}f(x)dx-\int_{a}^{b}f(x)dx=\int_{-1}^{6}f(x)dx\\
            &=\int_{-3}^{6}f(x)dx+\int_{b}^{a}f(x)dx\\
            a&=-3,\,\, b=-1
        \end{align}
    \item
        \begin{align}
            \set
            &\int_{a}^{b}\sin xdx<0=\int_{\pi}^{2\pi}\sin xdx<0\\
            a&=\pi,\,\, b=2\pi
        \end{align}
    \item
        \begin{align}
            \set
            &\int_{a}^{b}\cos xdx=0=\int_{0}^{\pi}\cos xdx\\
            a&=0,\,\, b=\pi
        \end{align}
\end{enumerate}

\section{Determine which value best approximates
the definite integral. Make your selection on the basis of a
sketch.}
58.\[\int_{0}^{1/2}4\cos\pi xdx\]
\begin{enumerate}[(a)]
    \item 4
    \item $\frac{4}{3}$
    \item 16
    \item $2\pi$
    \item -6
\end{enumerate}
The answer is (b)$A\approx\frac{4}{3}u^2$.

60.\[\int_{0}^{9}(1+\sqrt[]{x})dx\]
\begin{enumerate}[(a)]
    \item -3
    \item 9
    \item 27
    \item 3
\end{enumerate}
The answer is (c)$A\approx 27$.

\section{Determine whether the
statement is true or false. If it is false, explain why or give an
example that shows it is false.}
65.$\int_{a}^{b}(f(x)+g(x))dx=\int_{a}^{b}f(x)dx+\int_{a}^{b}g(x)dx$\\
\indent True.

\vs\next
66.$\int_{a}^{b}f(x)g(x)dx=\left(\int_{a}^{b}f(x)dx\right)\left(\int_{a}^{b}g(x)dx\right)$\\
\indent False because $\int_{0}^{1}x\sqrt[]{x}dx\neq\left(\int_{0}^{1}xdx\right)\left(\int_{0}^{1}\sqrt[]{x}dx\right)$.

\vs\next
67. If the norm of a partition approaches zero, then the number of
subintervals approaches infinity.\\
\indent True.

\vs\next
68. If $f$ is increasing on $[a, b]$, then the minimum value of $f(x)$ on $[a, b]$ is $f(a)$.\\
\indent True.

\vs\next
69. The value of $\int_{a}^{b}f(x)dx$ must be positive.\\
\indent False because $\int_{0}^{2}(-x)dx=-2$.

\vs\next
70. The value of $\int_{2}^{2}\sin(x^2)dx$ is 0.\\
\indent True.














\end{document}