\documentclass[11pt]{article}
\usepackage{amsmath, amssymb, amsfonts,  graphicx, enumerate, float, wrapfig, hyperref}
\usepackage[margin=0.5in]{geometry}
\graphicspath{{./}}
\newcommand*{\vs}{\vspace{1cm}}
\newcommand*{\next}{\noindent}
\newcommand*{\set}{\setcounter{equation}{0}}
\newcommand*{\im}{\includegraphics}
\newcommand*{\lt}{\left}
\newcommand*{\rt}{\right}

\begin{document}

\title{4.4 The Fundamental Theorem of Calculus}
\author{Juan J. Moreno Santos}
\date{December 2023}

\maketitle

\section{Use a graphing utility
to graph the integrand. Use the graph to determine whether the
definite integral is positive, negative, or zero.}
2.\begin{align}
    \set
    f(x)&=\cos x\\
    \int_{0}^{\pi}\cos xdx&=0
\end{align}

\vs\next
4.\begin{align}
    \set
    f(x)&=x\sqrt[]{2-x}\\
    \int_{-2}^{2}x\sqrt[]{2-x}dx&=negative
\end{align}

\section{Evaluate the definite integral of the algebraic
function. Use a graphing utility to verify your result.}
6.\begin{align}
    \set
    &\int_{4}^{9}5dv\\
    &=(5v)_4^9\\
    &=5(9)-5(4)=25
\end{align}

\vs\next
10.\begin{align}
    \set
    &\int_{1}^{7}(6x^2+2x-3)dx\\
    &=(2x^3+x^2-3x)_1^7\\
    &=(2(7)^3+(7)^2-3(7))-(2+1-3)=714
\end{align}

\vs\next
14.\begin{align}
    \set
    &\int_{-2}^{-1}\left(u-\frac{1}{u^2}\right)du\\
    &=\left(\frac{u^2}{2}+\frac{1}{u}\right)_-2^-1\\
    &=\left(\frac{1}{2}-1\right)-\left(2-\frac{1}{2}\right)=-2
\end{align}

\vs\next
18.\begin{align}
    \set
    &\int_{1}^{8}\sqrt[]{\frac{2}{x}}dx\\
    &=\sqrt[]{2}\int_{1}^{8}x^{-1/2}dx\\
    &=(\sqrt[]{2}(2)x^{1/2})_1^8\\
    &=(2\sqrt[]{2x})_1^8\\
    &=8-2\sqrt[]{2}
\end{align}

\vs\next
22.\begin{align}
    \set
    &\int_{-8}^{-1}\frac{x-x^2}{2\sqrt[3]{x}}dx\\
    &=\frac{1}{2}\int_{-8}^{-1}(x^{2/3}-x^{5/3})dx\\
    &=\frac{1}{2}\left(\frac{3}{5}x^{5/3}-\frac{3}{8}x^{8/3}\right)_{-8}^{-1}\\
    &=-\frac{1}{80}(39)+\frac{32}{80}(144)\\
    &=\frac{4569}{80}
\end{align}

\vs\next
26.\begin{align}
    \set
    &\int_{0}^{4}|x^2-4x+3|dx\\
    &=\int_{0}^{1}(x^2-4x+3)dx-\int_{1}^{3}(x^2-4x+3)dx+\int_{3}^{4}(x^2-4x+3)dx\\
    &=\left(\frac{x^3}{3}-2x^2+3x\right)_0^1-\left(\frac{x^3}{3}-2x^2+3x\right)_1^3+\left(\frac{x^3}{3}-2x^2+3x\right)_3^4\\
    &=\left(\frac{1}{3}-2+3\right)-(9-18+9)+\left(\frac{1}{3}-2+3\right)+\left(\frac{64}{3}-32+12\right)-(9-18+9)\\
    &=\frac{4}{3}+\frac{4}{3}+\frac{4}{3}=4
\end{align}

\section{Evaluate the definite integral of the trigonometric function. Use a graphing utility to verify your result.}
28.\begin{align}
    \set
    &\int_{0}^{\pi}(2+\cos x)dx\\
    &=(2x+\sin x)_0^{\pi}\\
    &=(2\pi+0)-0=2\pi
\end{align}

32.\begin{align}
    \set
    &\int_{\pi/4}^{\pi/2}(2-\csc^2x)dx\\
    &=(2x+\cot x)_{\pi/4}^{\pi/2}\\
    &=(\pi+0)-\left(\frac{\pi}{2}+1\right)\\
    &=\frac{\pi}{2}-1\\
    &=\frac{\pi-2}{2}
\end{align}

\vs\next
34.\begin{align}
    \set
    &\int_{-\pi/2}^{\pi/2}(2t+\cos t)dt\\
    &=(t^2+\sin t)_{-\pi/2}^{\pi/2}\\
    &=\left(\frac{\pi^2}{4}+1\right)-\left(\frac{\pi^2}{4}-1\right)\\
    &=2
\end{align}

\section{Determine the area of the given region.}
36.$y=\frac{1}{x^2}$\\
\im{36.png}\\
\begin{align}
    \set
    A&=\int_{1}^{2}\frac{1}{x^2}dx\\
    &=\left(-\frac{1}{x}\right)_1^2\\
    &=\frac{1}{2}
\end{align}

\vs\next
38.$y=x+\sin x$\\
\im{38.png}\\
\begin{align}
    \set
    A&=\int_{0}^{\pi}(x+\sin x)dx\\
    &=\left(\frac{x^2}{2}-\cos x\right)_0^{/pi}\\
    &=\frac{\pi^2}{2}+2\\
    &=\frac{\pi^2+4}{2}
\end{align}

\section{Find the area of the region bounded by the
graphs of the equations.}
40.\begin{align}
    \set
    y&=x^3+x,\,\, x=2,\,\, y=0\\
    A&=\int_{0}^{2}(x^3+x)dx\\
    &=\left(\frac{x^4}{4}+\frac{x^2}{2}\right)_0^2\\
    &=4+2=6
\end{align}

\vs\next
42.\begin{align}
    \set
    y&=(3-x)\sqrt[]{x},\,\, y=0\\
    A&=\int_{0}^{3}(3-x)\sqrt[]{x}dx\\
    &=\int_{0}^{3}(3x^{1/2}-x^{3/2})dx\\
    &=\left(\frac{3x^{3/2}}{\frac{3}{2}}-\frac{x^{5/2}}{\frac{5}{2}}\right)_0^3\\
    &=\lt(2x^{3/2}-\frac{2}{5}x^{5/2}\rt)_0^3\\
    &=2\cdot 3\cdot\sqrt[]{3}-\frac{2}{5}9\sqrt[]{3}\\
    &=\frac{12}{5}\sqrt[]{3}
\end{align}

\section{Find the value(s) of $c$ guaranteed by the
Mean Value Theorem for Integrals for the function over the
given interval.}
46.\begin{align}
    \set
    f(x)&=\frac{9}{x^3},\,\,[1, 3]\\
    \int_{1}^{3}\frac{9}{x^3}dx&=\lt(-\frac{9}{2x^2}\rt)_1^3\\
    &=-\frac{1}{2}+\frac{9}{2}=4\\
    f(c)(3-1)&=4\\
    \frac{9}{c^3}&=2\\
    c^3&=\frac{9}{2}\\
    c&=\sqrt[3]{\frac{9}{2}}\approx 1.6510
\end{align}

\vs\next
50.\begin{align}
    \set
    f(x)&=\cos x,\,\, (-\frac{\pi}{3}, \frac{\pi}{3})\\
    \int_{-\pi/3}^{\pi/3}\cos xdx&=(\sin x)_{-\pi/3}^{\pi/3}=\sqrt[]{3}\\
    f(c)\lt(\frac{\pi}{3}-\lt(-\frac{\pi}{3}\rt)\rt)&=\sqrt[]{3}\\
    \cos c&=\frac{3\sqrt[]{3}}{2\pi}\\
    c&\approx\pm 0.5971
\end{align}

\section{Find the average value of the function over
the given interval and all values of $x$ in the interval for which the
function equals its average value.}
52.\begin{align}
    \set
    f(x)&=\frac{4(x^2+1)}{x^2},\,\,[1, 3]\\
    \frac{1}{3-1}\int_{1}^{3}\frac{4(x^2+1)}{x^2}dx&=2\int_{1}^{3}(1+x^{-2})dx\\
    &=2\lt(x-\frac{1}{x}\rt)_1^3\\
    &=2\lt(3-\frac{1}{3}\rt)=\frac{16}{3}\\
    \frac{4(x^2+1)}{x^2}&=\frac{16}{3}\Rightarrow x=\sqrt[]{3},\,\,[1, 3]
\end{align}

\vs\next
56.\begin{align}
    \set
    f(x)&=\cos x,\,\,[0,\frac{\pi}{2}]\\
    &\frac{1}{\lt(\frac{\pi}{2}\rt)-0}\int_{0}^{\pi/2}\cos xdx\\
    &=\lt(\frac{2}{\pi}\sin x\rt)_0^{\pi/2}\\
    &=\frac{2}{\pi}\\
    \cos x&=\frac{2}{\pi}\\
    x&\approx 0.881
\end{align}

\section{Velocity}
58.The graph shows the velocity, in feet per second, of a
decelerating car after the driver applies the brakes. Use the graph
to estimate how far the car travels before it comes to a stop.\\
\im{58.png}\\
The traveled distance is $\int_{0}^{5}v(t)dt$. The area under the curve at $0\leq t\leq 5$ is about $29(5)=145\text{ft}^2$.

\section{Average sales}
64. A company fits a model to the monthly sales
data for a seasonal product. The model is $S(t)=\frac{t}{4}+1.8+0.5\sin\lt(\frac{\pi t}{6}\rt),\,\, 0\leq t\leq 24$ where $S$ is sales (in 
thousands) and $t$ is time in months.

\begin{enumerate}[(a)]
    \item Use a graphing utility to graph $f(t)=0.5\sin\lt(\frac{\pi t}{6}\rt)$ for $0\leq t\leq 24$. Use the graph to explain why the average value of $f(t)$ is 0 over the interval.\\
        \indent The area above the x-axis is the same as the area below the x-axis. Therefore, the average value is 0.
    \item Use a graphing utility to graph $S(t)$ and the line $g(t)=\frac{t}{4}+1.8$ in the same viewing window. Use the graph and the result of part (a) to explain why $g$ is called the $trend$ $line$.
        \indent $g$ is called the $trend$ $line$ because the average value of $S$ approaches this line.
\end{enumerate}

\section{Capstone}
66. The graph of $f$ is shown in the figure. The shaded region $A$ has an area of 1.5, and $\int_{0}^{6}f(x)dx=3.5$. Use this information to fill in the blanks.
\begin{enumerate}[(a)]
    \item $\int_{0}^{2}f(x)dx=-1.5$
    \item $\int_{2}^{6}f(x)dx=5$
    \item $\int_{0}^{6}|f(x)|dx=6.5$
    \item $\int_{0}^{2}-2f(x)dx=3$
    \item $\int_{0}^{6}(2+f(x))dx=15.5$
    \item The average value of $f$ over the interval [0. 6] is 0.5833.
\end{enumerate}

\section{Find F as a function of $x$ and evaluate it at $x=2$, $x=5$, and $x=8$.}
68.\begin{align}
    \set
    F(x)&=\int_{2}^{x}(t^3+2t-2)dt\\
    &=\lt(\frac{t^4}{4}+t^2-2t\rt)_2^x\\
    &=\lt(\frac{x^4}{4}+x^2-2x\rt)-(4+4-4)\\
    &=\frac{x^4}{4}+x^2-2x-4\\
    F(2)&=4+4-4-4=0\\
    F(5)&=\frac{625}{4}+25-10-4=167.25\\
    F(8)&=\frac{8^4}{4}+64-16-4=1068
\end{align}

\vs\next
72.\begin{align}
    \set
    F(x)&=\int_{0}^{x}\sin\theta d\theta=-\cos\theta|_0^x\\
    &=-\cos x+\cos 0\\
    &=1-\cos x\\
    F(2)&=1-\cos 2\approx 1.4161\\
    F(5)&=1-\cos 5\approx 0.7163\\
    F(8)&=1-\cos 8\approx 1.1455
\end{align}

\vs\next
74. Let $g(x)=\int_{0}^{x}f(t)dt$, where $f$ is the function whose graph is shown in the figure.\\
\im{74.png}\\
\begin{enumerate}[(a)]
    \item Estimate $g(0)$, $g(2)$, $g(4)$, $g(6)$, and $g(8)$
        \begin{align}
            \set
            g(0)&=\int_{0}^{0}f(t)dt=0\\
            g(2)&=\int_{0}^{2}f(t)dt=-\frac{1}{2}(2)(4)=-4\\
            g(4)&=\int_{0}^{4}f(t)dt=-\frac{1}{2}(4)(4)=-8\\
            g(6)&=\int_{0}^{6}f(t)dt=-8+2+4=-2\\
            g(8)&=\int_{0}^{8}f(t)dt=-2+6=4
        \end{align}
    \item Find the largest open interval on which $g$ is increasing. Find the largest open interval on which $g$ is decreasing.\\
        \indent $g$ is decreasing on (0, 4) and increasing on (4, 8)
    \item Identify any extrema of $g$.\\
        \indent g is a minimum of -8 at $x=4$\\
\end{enumerate}

\section{(a) Integrate to find $F$ as a function of $x$ and
(b) demonstrate the Second Fundamental Theorem of Calculus
by differentiating the result in part (a).}
76.\begin{enumerate}[(a)]
    \item
        \begin{align}
            \set
            \int_{0}^{x}t(t^2+1)dt&=\int_{0}^{x}(t^3+t)dt\\
            &=\lt(\frac{1}{4}t^4+\frac{1}{2}t^2\rt)|_0^x\\
            &=\frac{1}{4}x^4+\frac{1}{2}x^2\\
            &=\frac{x^2}{4}(x^2+2)   
        \end{align}
    \item
        \begin{align}
            \set
            &\frac{d}{dx}\lt(\frac{1}{4}x^4+\frac{1}{2}x^2\rt)\\
            &=x^3+x\\
            &=x(x^2+1)
        \end{align}
\end{enumerate}

\section{Use the Second Fundamental Theorem of
Calculus to find $F'(x)$}
84.\begin{align}
    \set
    F(x)&=\int_{1}^{x}\sqrt[4]{t}dt\\
    F'(x)&=\sqrt[4]{x}
\end{align}

\section{Find $F'(x)$.}
88.\begin{align}
    \set
    F(x)&=\int_{-x}^{x}t^3dt=\lt(\frac{t^4}{4}\rt)_{-x}^x=0\\
    F'(x)&=0
\end{align}

\vs\next
92.\begin{align}
    \set
    F(x)&=\int_{0}^{x^2}\sin\theta^2 d\theta\\
    F'(x)&=\sin(x^2)^2(2x)\\
    &=2x\sin x^4
\end{align}

\vs\next
94. Use the graph of the function $f$ show in the figure and the function called $g$ defined by $g(x)=\int_{0}^{x}f(t)dt$.\\
\im{94.png}\\
\begin{enumerate}[(a)]
\item Complete the table.
\begin{flushleft}
    \begin{table}[h]
        \begin{tabular}{|l|l|l|l|l|l|l|l|l|l|l|}
        \hline
         $x$ & 1 & 2 & 3 & 4 & 5 & 6 & 7 & 8 & 9 & 10\\\hline
         $g(x)$ & 1 & 2 & 0 & -2 & -4 & -6 & -3 & 0 & 3 & 6\\\hline
        \end{tabular}
    \end{table}
\end{flushleft}
\item Where does $g$ have its minimum? Explain.\\
    \indent $g$ is minimum at (6, -6).
\item Where does $g$ have a maximum? Explain.\\
    \indent $g$ has a relative maximum at (2, 2).
\item On what interval does $g$ increase at the greatest rate? Explain.\\
    \indent $g$ increases at a rate of 3 on [6, 10].
\item Identify the zeros of $g$.\\
    \indent The zeros of $g$ are $x=3,\,\, 8$.
\end{enumerate}

\section{The velocity function, in feet per second, is
given for a particle moving along a straight line. Find (a) the
displacement and (b) the total distance that the particle travels
over the given interval.}
101.\begin{enumerate}[(a)]
    \item\begin{align}
        \set
        v(t)&=\frac{1}{\sqrt[]{t}},\,\, 1\leq t\leq 4\\
        \text{Total Distance}&=\text{Displacement}\\
        \text{Displacement}&=\int_{1}^{4}t^{-1/2}dt\\
        &=\lt(2t^{1/2}\rt)_1^4\\
        &=4-2=2\text{ft to the right}
    \end{align}
    \item The total distance is 2 feet.
\end{enumerate}

\vs\next
102.\begin{enumerate}[(a)]
    \item\begin{align}
        \set
        v(t)&=\cos t,\,\, 0\leq t\leq 3\pi\\
        \text{Displacement}&=\int_{0}^{3\pi}\cos tdt\\
        &=\lt(\sin t\rt)_0^{3\pi}=0\text{ft}\\
    \end{align}
    \item\begin{align}
        \set
        \text{Total Distance}&=\int_{0}^{\pi/2}\cos tdt-\int_{\pi/2}^{3\pi/2}\cos tdt+\int_{3\pi/2}^{5\pi/2}\cos tdt-\int_{5\pi/2}^{3\pi}\cos tdt\\
        &=(\sin t)_0^{\pi/2}-(\sin t)_{\pi/2}^{3\pi/2}+(\sin t)_{3\pi/2}^{5\pi/2}-(\sin t)_{5\pi/2}^{3\pi}\\
        &=1-(-2)+2-(-1)=6
    \end{align}
\end{enumerate}

\section{Oil Leak}
106. At 1:00 P.M., oil begins leaking from a tank at a
rate of $4+0.75t$ gallons per hour.
\begin{enumerate}[(a)]
    \item How much oil is lost from 1:00 P.M. to 4:00 P.M.?
        \begin{align}
            \set
            &\int_{0}^{3}(4+0.75t)dt\\
            &=\lt(4t+\frac{0.75}{2}t^2\rt)_0^3\\
            &=\frac{123}{8}=15.375\text{gal}
        \end{align}
    \item How much oil is lost from 4:00 P.M. to 7:00 P.M.?
        \begin{align}
            \set
            &\int_{3}^{6}(4+0.75t)dt\\
            &=\lt(4t+\frac{0.75}{2}t^2\rt)_3^6\\
            &=22.125\text{gal}
        \end{align}
    \item Compare your answers from parts (a) and (b). What do
    you notice?
        \indent The oil lost in the evening is more because the flow rate increased.    
\end{enumerate}

\section{Determine whether
the statement is true or false. If it is false, explain why or give
an example that shows it is false.}
113. If $F'(x)=G'(x)$ on the interval [a, b], then $F(b)-F(a)=G(b)-G(a)$.\\
\indent True.

\vs\next
114. If $f$ is continuous on [a, b], then $f$ is integrable on [a, b].\\
\indent True.




























































\end{document}