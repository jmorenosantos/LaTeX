\documentclass[11pt]{article}
\usepackage{amsmath, amssymb, amsfonts,  graphicx, enumerate, float, wrapfig, hyperref}
\usepackage[margin=0.5in]{geometry}
\graphicspath{{./}}
\newcommand*{\vs}{\vspace{1cm}}
\newcommand*{\next}{\noindent}
\newcommand*{\set}{\setcounter{equation}{0}}
\newcommand*{\im}{\includegraphics}
\newcommand*{\lt}{\left}
\newcommand*{\rt}{\right}

\begin{document}

\title{4.5 Integration by Substitution}
\author{Juan J. Moreno Santos}
\date{December 2023}

\maketitle

\section{Complete the table by identifying and for
the integral.}
2.\begin{align}
    \set
    &\int x^2\sqrt[]{x^3+1}dx\\
    u&=x^3+1\\
    du&=3x^2dx
\end{align}

\vs\next
4.\begin{align}
    \set
    &\int\sec 2x\tan 2xdx\\
    u&=2x\\
    du&=2dx
\end{align}

\section{Determine whether it is necessary to use substitution to evaluate the integral. (Do not evaluate the integral.)}
8.\[\int x\sqrt[]{x+4}dx\]
Substitution is necessary. $u=x+4$

\vs\next
10.\[\int x\cos x^2dx\]
Substitution is necessary. $u=x^2$

\section{Find the indefinite integral and check the
result by differentiation.}
12.\begin{align}
    \set
    \int(x^2-9)^3(2x)dx&=\frac{(x^2-9)^4}{4}+C\\
    \frac{d}{dx}\lt(\frac{(x^2-9)^4}{4}+C\rt)&=\frac{4(x^2-9)^3}{4}(2x)\\
    &=(x^2-9)^3(2x)
\end{align}

\vs\next
14.\begin{align}
    \set
    \int\sqrt[3]{3-4x^2}(-8x)dx&=\int(3-4x^2)^{1/3}(-8x)dx\\
    &=\frac{(3-4x^2)^{4/3}}{\frac{4}{3}}+C\\
    &=\frac{3}{4}(3-4x^2)^{4/3}+C\\
    \frac{d}{dx}\lt(\frac{3}{4}(3-4x^2)^{4/3}+C\rt)&=\frac{3}{4}\lt(\frac{4}{3}\rt)(3-4x^2)^{1/3}(-8x)\\
    &=(3-4x^2)^{1/3}(-8x)
\end{align}

\vs\next
16.\begin{align}
    \set
    \int x^2(x^3+5)^4dx&=\frac{1}{3}\int(x^3+5)^4(3x^2)dx\\
    &=\frac{1}{3}\frac{(x^3+5)^5}{5}+C\\
    &=\frac{(x^3+5)^5}{15}+C\\
    \frac{d}{dx}\lt(\frac{(x^3+5)^5}{15}+C\rt)&=\frac{5(x^3+5)^4(3x^2)}{15}\\
    &=(x^3+5)^4x^2
\end{align}

\vs\next
20.\begin{align}
    \set
    \int t^3\sqrt[]{t^4+5}dt&=\frac{1}{4}\int(t^4+5)^{1/2}(4t^3)dt\\
    &=\frac{1}{4}\frac{(t^4+5)^{3/2}}{\frac{3}{2}}+C\\
    &=\frac{1}{6}(t^4+5)^{3/2}+C\\
    \frac{d}{dt}\lt(\frac{1}{6}(t^4+5)^{3/2}+C\rt)&=\frac{1}{6}\cdot\frac{3}{2}(t^4+5)^{1/2}(4t^3)\\
    &=(t^4+5)^{1/2}(t^3)
\end{align}

\vs\next
30.\begin{align}
    \set
    \int\lt(x^2+\frac{1}{(3x)^2}\rt)dx&=\int\lt(x^2+\frac{1}{9}x^{-2}\rt)dx\\
    &=\frac{x^3}{3}+\frac{1}{9}\lt(\frac{x^{-1}}{-1}\rt)+C\\
    &=\frac{x^3}{3}-\frac{1}{9x}+C\\
    &=\frac{3x^4-1}{9x}+C\\
    \frac{d}{dx}\lt(\frac{1}{3}x^3-\frac{1}{9}x^{-1}+C\rt)&=x^2+\frac{1}{9}x^{-2}\\
    &=x^2+\frac{1}{(3x)^2}
\end{align}

\vs\next
34.\begin{align}
    \set
    \int\frac{t-9t^2}{\sqrt[]{t}}dt&=\int(t^{1/2}-9t^{3/2})dt\\
    &=\frac{2}{3}t^{3/2}-\frac{18}{5}t^{5/2}+C\\
    &=\frac{2}{15}^{3/2}(5-27t)+C\\
    \frac{d}{dt}\lt(\frac{2}{3}t^{3/2}-\frac{18}{5}t^{5/2}+C\rt)&=t^{1/2}-9t^{3/2}\\
    &=\frac{t-9t^2}{\sqrt[]{t}}
\end{align}

\vs\next
38.\begin{align}
    \set
    \int 4\pi y(6+y^{3/2})dy&=\int(24\pi+4\pi y^{5/2})dy\\
    &=12\pi y^2+\frac{8\pi}{7}y^{7/2}+C\\
    \frac{d}{dy}\lt(12\pi y^2+\frac{8\pi}{7}y^{7/2}+C\rt)&=24\pi y+4\pi y^{5/2}\\
    &=4\pi y(6+y^{3/2})
\end{align}

\section{Solve the differential equation.}
40.\begin{align}
    \set
    \frac{dy}{dx}&=\frac{10x^2}{\sqrt[]{1+x^3}}\\
    y&=\int\frac{10x^2}{\sqrt[]{1+x^3}}dx\\
    &=\frac{10}{3}\int(1+x^3)^{-1/2}(3x^2)dx\\
    &=\frac{10}{3}\lt(\frac{(1+x^3)^{1/2}}{1/2}\rt)+C\\
    &=\frac{20}{3}\sqrt[]{1+x^3}+C
\end{align}

\section{A differential equation, a
point, and a slope field are given. A $slope field$ consists of line
segments with slopes given by the differential equation. These line
segments give a visual perspective of the directions of the
solutions of the differential equation. (a) Sketch two approximate
solutions of the differential equation on the slope field, one of
which passes through the given point. (b) Use
integration to find the particular solution of the differential
equation and use a graphing utility to graph the solution.
Compare the result with the sketches in part (a).}
44.\[\frac{dy}{dx}=x^2(x^3-1)^2,\,\, (1, 0)\]
\im{44a.png}
\begin{align}
    \set
    \frac{dy}{dx}&=x^2(x^3-1)^2,\,\, (1, 0)\\
    y&=\int x^2(x^3-1)^2dx\\
    (u=x^3-1)&=\frac{1}{3}\int(x^3-1)^2(3x^2dx)\\
    &=\frac{1}{3}\frac{(x^3-1)^3}{3}+C=\frac{1}{9}(x^3-1)^3+C\\
    0&=C\\
    y&=\frac{1}{9}(x^3-1)^3
\end{align}

\section{Find the indefinite integral.}
48.\begin{align}
    \set
    &\int 4x^3\sin x^4dx\\
    &=\int\sin x^4(4x^3)dx\\
    &=-\cos x^4+C
\end{align}

52.\begin{align}
    \set
    &\int x\sin x^2dx\\
    &=\frac{1}{2}\int(\sin x^2)(2x)dx\\
    &=-\frac{1}{2}\cos x^2+C
\end{align}

56.\begin{align}
    \set
    &\int\sqrt[]{tanx}\sec^2xdx\\
    &=\frac{(\tan x)^{3/2}}{\frac{3}{2}}+C\\
    &=\frac{2}{3}(\tan x)^{3/2}+C
\end{align}

60.\begin{align}
    \set
    &\int\csc^2\lt(\frac{x}{2}\rt)dx\\
    &=2\int\csc^2\lt(\frac{x}{2}\rt)\lt(\frac{1}{2}\rt)dx\\
    &=-2\cot\lt(\frac{x}{2}\rt)+C
\end{align}

\section{Find an equation for the function that has
the given derivative and whose graph passes through the given
point.}
62.\begin{align}
    \set
    f'(x)&=\pi\sec\pi x\tan\pi x,\,\, (\frac{1}{3}, 1)\\
    f(x)&=\int\pi\sec\pi x\tan\pi xdx\\
    &=\sec\pi x+C\\
    f(x)&=\sec\pi x-1\because f\lt(\frac{1}{3}\rt)=1
\end{align}

66.\begin{align}
    \set
    f'(x)&=-2x\sqrt[]{8-x^2},\,\, (2, 7)\\
    f(x)&=\frac{2(8-x^2)^{3/2}}{3}+C\\
    f(2)&=\frac{2(4)^{3/2}}{3}+C\\
    &=\frac{16}{3}+C\\
    &=7\Rightarrow C=\frac{5}{3}
\end{align}

\section{Evaluate the definite integral. Use a graphing
utility to verify your result.}
76.\begin{align}
    \set
    &\int_{-2}^{4}x^2(x^3+8)^2dx,\,\, u=x^3+8,\,\, du=3x^2dx\\
    &=\frac{1}{3}\int_{-2}^{4}(x^3+8)^2(3x^2)dx\\
    &=\lt(\frac{1}{3}\frac{(x^3+8)^3}{3}\rt)_{-2}^4\\
    &=\frac{1}{9}\lt((64+8)^3-(-8+8)^3\rt)=41.472
\end{align}

\vs\next
82.\begin{align}
    \set
    &\int_{0}^{2}x\sqrt[3]{4+x^2}dx,\,\, u=4+x^2,\,\, du=2xdx\\
    &=\frac{1}{2}\int_{0}^{2}(4+x^2)^{1/3}(2x)dx\\
    &=\lt(\frac{3}{8}(4+x^2)^{4/3}\rt)_0^2\\
    &=\frac{3}{8}(8^{4/3}-4^{4/3})\\
    &=6-\frac{3}{2}\sqrt[3]{4}\approx 3.619
\end{align}

\vs\next
86.\begin{align}
    \set
    &\int_{\pi/3}^{\pi/2}(x+\cos x)dx\\
    &=\lt(\frac{x^2}{2}+\sin x\rt)_{\pi/3}^{\pi/2}\\
    &=\lt(\frac{\pi^2}{8}+1\rt)-\lt(\frac{\pi^2}{18}+\frac{\sqrt[]{3}}{2}\rt)\\
    &=\frac{5\pi^2}{72}+\frac{2-\sqrt[]{3}}{2}
\end{align}

\section{The graph of a
function $f$ is shown. Use the differential equation and the given
point to find an equation of the function.}
88.\\\im{88.png}\begin{align}
    \set
    \frac{dy}{dx}&=\frac{-48}{(3x+5)^3},\,\, (-1, 3)\\
    y&=-48\int(3x+5)^{-3}dx\\
    &=(-48)\frac{1}{3}\int(3x+5)^{-3}dx\\
    &=\frac{-16(3x+5)^{-2}}{-2}+C\\
    3&=\frac{8}{(3(-1)+5)^2}+C\\
    &=\frac{8}{4}+C\Rightarrow 1\\
    y&=\frac{8}{(3x+5)^2}+1
\end{align}

\vs\next
90.\\\im{90.png}\begin{align}
    \set
    \frac{dy}{dx}&=4x+\frac{9x^2}{(3x^3+1)^{3/2}},\,\, (0, 2)\\
    y&=\int(4x+(3x^3+1)^{-3/2}9x^2)dx\\
    &=2x^2-\frac{2}{\sqrt[]{3x^3+1}}+C\\
    2&=0=\frac{2}{1}+C\Rightarrow C=4\\
    y&=2x^2-\frac{2}{\sqrt[]{3x^3+1}}+4
\end{align}

\section{Find the area of the region. Use a graphing
utility to verify your result.}
94\\\im{94.png}\begin{align}
    \text{Area}&=\int_{0}^{\pi}(\sin x+\cos 2x)dx\\
    &=\lt(-\cos x+\frac{1}{2}\sin 2x\rt)_0^{\pi}\\
    &=2
\end{align}

\section{Use a graphing utility to evaluate the
integral. Graph the region whose area is given by the definite
integral.}
98.\begin{align}
    \set
    &\int_{0}^{2}x^3\sqrt[]{2x+3}dx\\
    &\approx 9.945
\end{align}

\vs\next
100.\begin{align}
    \set
    &\int_{1}^{5}x^2\sqrt[]{x-1}dx\\
    &\approx 67.505
\end{align}

\vs\next
102.\begin{align}
    \set
    &\int_{0}^{\pi/6}\cos 3xdx\\
    &=\frac{1}{3}
\end{align}

\section{Evaluate the integral using the properties
of even and odd functions as an aid.}
104.\begin{align}
    \set
    f(x)&=x(x^2+1)^3\therefore\,\,\text{odd}\\
    \int_{-2}^{2}x(x^2+1)^3dx&=0
\end{align}

\vs\next
106.\begin{align}
    \set
    f(x)&=\sin x\cos x\therefore \text{odd}\\
    \int_{-\pi/2}^{\pi/2}\sin x\cos xdx&=0
\end{align}

\vs\next
108.\section{Use the symmetry of the graphs of the sine and cosine
functions as an aid in evaluating each definite integral.}
\begin{enumerate}[(a)]
    \item $\int_{-\pi/4}^{\pi/4}\sin xdx=0$ because $\sin x$ is symmetric to the origin.
    \item $\int_{-\pi/4}^{\pi/4}\cos xdx=2\int_{0}^{\pi/4}\cos xdx=(2\sin x)_0^{\pi/4}=\sqrt[]{2}$ because $\cos x$ is symmetric to the $y$-axis
    \item $\int_{-\pi/2}^{\pi/2}\cos xdx=2\int_{0}^{\pi/2}\cos xdx=(2\sin x)_0^{\pi/2}=2$
    \item $\int_{-\pi/2}^{\pi/2}\sin x\cos xdx=0$ because $\sin(-x)\cos(-x)=-\sin x\cos x$, making it symmetric to the origin.
\end{enumerate}

\vs\next
116 can't be completed due to a formatting error in the textbook (the initial and decreasing values aren't specified).\\
\im[scale=0.5]{116.png}\\

\section{Sales}
118. The sales $S$ (in thousands of units) of a seasonal
product are given by the model \[S=74.50+43.75\sin\frac{\pi t}{6}\] where $t$ is the time in months, with $t=1$ corresponding to January. Find the average sales for each time period.
\begin{enumerate}[(a)]
    \item The first quarter $(0\leq t\leq 3)$
        \begin{align}
            \set
            \frac{1}{b-a}\int_{a}^{b}\lt(74.5+43.75\sin\frac{\pi t}{6}\rt)dt&=\frac{1}{b-a}\lt(74.5t-\frac{262.5}{\pi}\cos\frac{\pi t}{6}\rt)_a^b\\
            \frac{1}{3}\lt(74.5t-\frac{262.5}{\pi}\cos\frac{\pi t}{6}\rt)_0^3&=\frac{1}{3}\lt(223.5+\frac{262.5}{\pi}\rt)\approx 102.352\,\,\text{thousand units}
        \end{align}    
    \item The second quarter $(3\leq t\leq 6)$
        \begin{align}
            \set
            \frac{1}{3}\lt(74.5t-\frac{262.5}{\pi}\cos\frac{\pi t}{6}\rt)_3^6&=\frac{1}{3}\lt(447+\frac{262.5}{\pi}-223.5\rt)\approx 102.352\,\,\text{thousand units}
        \end{align}
    \item The entire year $(0\leq t\leq 12)$
        \begin{align}
            \set
            \frac{1}{12}\lt(74.5t-\frac{262.5}{\pi}\cos\frac{\pi t}{6}\rt)_0^{12}&=\frac{1}{12}\lt(894-\frac{262.5}{\pi}+\frac{262.5}{\pi}\rt)\approx 74.5\,\,\text{thousand units}
        \end{align}
\end{enumerate}

\section{Determine whether the
statement is true or false. If it is false, explain why or give an
example that shows it is false.}
129.$\int(2x+1)^2dx=\frac{1}{3}(2x+1)^3+C$\\
\indent False because $\int(2x+1)^2dx=\frac{1}{2}\int(2x+1)^2 2dx=\frac{1}{6}(2x+1)^3+C$.

\vs\next
130.$\int x(x^2+1)dx=\frac{1}{2}x^2\lt(\frac{1}{3}x^3+x\rt)+C$\\
\indent False because $\int x(x^2+1)dx=\frac{1}{2}\int(x^2+1)(2x)dx=\frac{1}{4}(x^2+1)^2+C$

\vs\next
131.$\int_{-10}^{10}(ax^3+bx^2+cx+d)dx=2\int_{0}^{10}(bx^2+d)dx$\\
\indent True.

\vs\next
132.$\int_{a}^{b}\sin xdx=\int_{a}^{b+2\pi}\sin xdx$\\
\indent True.

\vs\next
133.$4\int\sin x\cos xdx=-\cos 2x+C$\\
\indent True.

\vs\next
134.$\int\sin^2 2x\cos 2xdx=\frac{1}{3}\sin^3 2x+C$\\
\indent False.
    \begin{align}
        \set
        \int\sin^2 2x\cos 2xdx&=\frac{1}{2}\int(\sin 2x)^2(2\cos 2x)dx\\
        &=\frac{1}{2}\frac{(\sin 2x)^3}{3}+C\\
        &=\frac{1}{6}\sin^3 2x+C
    \end{align}



































\end{document}