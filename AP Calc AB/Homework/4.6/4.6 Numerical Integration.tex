\documentclass[11pt]{article}
\usepackage{amsmath, amssymb, amsfonts,  graphicx, enumerate, float, wrapfig, hyperref}
\usepackage[margin=0.5in]{geometry}
\graphicspath{{./}}
\newcommand*{\vs}{\vspace{1cm}}
\newcommand*{\next}{\noindent}
\newcommand*{\set}{\setcounter{equation}{0}}
\newcommand*{\im}{\includegraphics}
\newcommand*{\lt}{\left}
\newcommand*{\rt}{\right}

\begin{document}

\title{4.6 Numerical Integration}
\author{Juan J. Moreno Santos}
\date{December 2023}

\maketitle
\section{Use the Trapezoidal Rule and Simpson’s Rule
to approximate the value of the definite integral for the given
value of Round your answer to four decimal places and
compare the results with the exact value of the definite integral.}
Equations (1), (2), and (3) will be the exact value, the Trapezoidal Rule, and Simpson's Rule respectively.\\
2.\begin{align}
    \set
    \int_{1}^{2}\lt(\frac{x^2}{4}+1\rt)dx&=\lt(\frac{x^3}{12}+x\rt)_1^2=\frac{19}{12}\approx 1.5833\\
    \int_{1}^{2}\lt(\frac{x^2}{4}+1\rt)dx&\approx\frac{1}{8}\lt(\lt(\frac{1^2}{4}+1\rt)+2\lt(\frac{\lt(\frac{5}{4}\rt)^2}{4}+1\rt)+2\lt(\frac{\lt(\frac{3}{2}\rt)^2}{4}+1\rt)+2\lt(\frac{\lt(\frac{7}{4}\rt)^2}{4}+1\rt)+\lt(\frac{2^2}{4}+1\rt)\rt)=\frac{203}{128}\approx 1.5859\\
    \int_{0}^{1}\lt(\frac{x^2}{4}+1\rt)dx&\approx\frac{1}{12}\lt(\lt(\frac{1^2}{4}+1\rt)+4\lt(\frac{\lt(\frac{5}{4}\rt)^2}{4}+1\rt)+2\lt(\frac{\lt(\frac{3}{2}\rt)^2}{4}+1\rt)+4\lt(\frac{\lt(\frac{7}{4}\rt)^2}{4}+1\rt)+\lt(\frac{2^2}{4}+1\rt)\rt)=\frac{19}{12}\approx 1.5833
\end{align}

\vs\next
6.\begin{align}
    \set
    \int_{0}^{8}\sqrt[3]{x}dx&=\lt(\frac{3}{4}x^{4/3}\rt)_0^8=12\\
    \int_{0}^{8}\sqrt[3]{x}dx&\approx\frac{1}{2}(0+2+2\sqrt[3]{2}+2\sqrt[3]{3}+2\sqrt[3]{4}+2\sqrt[3]{5}+2\sqrt[3]{6}+2\sqrt[3]{7}+2)\approx 11.7296\\
    \int_{0}^{8}\sqrt[3]{x}dx&\approx\frac{1}{3}(0+4+2\sqrt[3]{2}+4\sqrt[3]{3}+2\sqrt[3]{4}+4\sqrt[3]{5}+2\sqrt[3]{6}+4\sqrt[3]{7}+2)\approx 11.8632
\end{align}

\vs\next
10.\begin{align}
    \set
    \int_{0}^{2}x\sqrt[]{x^2+1}dx&=\frac{1}{3}((x^2+1)^{3/2})_0^2=\frac{1}{3}(5^{3/2}-1)\approx 3.393\\
    \int_{0}^{2}x\sqrt[]{x^2+1}dx&\approx\frac{1}{4}\lt(0+2\lt(\frac{1}{2}\rt)\sqrt[]{\lt(\frac{1}{2}\rt)^2+1}+2\lt(1\rt)\sqrt[]{\lt(1\rt)^2+1}+2\lt(\frac{3}{2}\rt)\sqrt[]{\lt(\frac{3}{2}\rt)^2+1}+2\sqrt[]{\lt(2\rt)^2+1}\rt)\approx 3.457\\
    \int_{0}^{2}x\sqrt[]{x^2+1}dx&\approx\frac{1}{6}\lt(0+4\lt(\frac{1}{2}\rt)\sqrt[]{\lt(\frac{1}{2}\rt)^2+1}+2\lt(1\rt)\sqrt[]{\lt(1\rt)^2+1}+4\lt(\frac{3}{2}\rt)\sqrt[]{\lt(\frac{3}{2}\rt)^2+1}+2\sqrt[]{\lt(2\rt)^2+1}\rt)\approx 3.392\
\end{align}

\section{Approximate the definite integral using the
Trapezoidal Rule and Simpson’s Rule with $n=4$. Compare
these results with the approximation of the integral using a
graphing utility.}
Equations (1), (2), and (3) will be the exact value, the Trapezoidal Rule, and Simpson's Rule respectively.\\
12.\begin{align}
    \set
    \int_{0}^{2}\frac{1}{\sqrt[]{1+x^3}}dx&=1.402\\
    \int_{0}^{2}\frac{1}{\sqrt[]{1+x^3}}dx&\approx\frac{1}{4}\lt(1+2\lt(\frac{1}{\sqrt[]{1+\lt(\frac{1}{2}\rt)^3}}\rt)+2\lt(\frac{1}{\sqrt[]{1+\lt(1\rt)^3}}\rt)+2\lt(\frac{1}{\sqrt[]{1+\lt(\frac{3}{2}\rt)^3}}\rt)+\frac{1}{3}\rt)\approx 1.397\\
    \int_{0}^{2}\frac{1}{\sqrt[]{1+x^3}}dx&\approx\frac{1}{6}\lt(1+4\lt(\frac{1}{\sqrt[]{1+\lt(\frac{1}{2}\rt)^3}}\rt)+2\lt(\frac{1}{\sqrt[]{1+\lt(1\rt)^3}}\rt)+4\lt(\frac{1}{\sqrt[]{1+\lt(\frac{3}{2}\rt)^3}}\rt)+\frac{1}{3}\rt)\approx 1.405
\end{align}

16.\begin{align}
    \set
    \int_{0}^{\sqrt[]{\frac{\pi}{2}}}\tan x^2dx&=0.256\\
    \int_{0}^{\sqrt[]{\frac{\pi}{2}}}\tan x^2dx&\approx\frac{\sqrt[]{\frac{\pi}{4}}}{8}\lt(\tan 0+2\tan\lt(\frac{\sqrt[]{\frac{\pi}{4}}}{4}\rt)^2+2\tan\lt(\frac{\sqrt[]{\frac{\pi}{4}}}{2}\rt)^2+2\tan\lt(\frac{\sqrt[]{\frac{3\pi}{4}}}{4}\rt)^2+\tan\lt(\sqrt[]{\frac{\pi}{4}}\rt)^2\rt)\approx 0.271\\
    \int_{0}^{\sqrt[]{\frac{\pi}{2}}}\tan x^2dx&\approx\frac{\sqrt[]{\frac{\pi}{4}}}{12}\lt(\tan 0+4\tan\lt(\frac{\sqrt[]{\frac{\pi}{4}}}{4}\rt)^2+2\tan\lt(\frac{\sqrt[]{\frac{\pi}{4}}}{2}\rt)^2+4\tan\lt(\frac{\sqrt[]{\frac{3\pi}{4}}}{4}\rt)^2+\tan\lt(\sqrt[]{\frac{\pi}{4}}\rt)^2\rt)\approx 0.257
\end{align}

\vs\next
20.\begin{align}
    \set
    \int_{0}^{\pi}f(x)dx&=1.852,\,\, f(x)=\begin{cases}
        \frac{\sin x}{x},\,\, x>0\\
        1,\,\, x=0
        \end{cases}\\
    \int_{0}^{\pi}\frac{\sin x}{x}dx&\approx\frac{\pi}{8}\lt(1+\frac{2\sin\lt(\frac{\pi}{4}\rt)}{\frac{\pi}{4}}+\frac{2\sin\lt(\frac{\pi}{2}\rt)}{\frac{\pi}{2}}+\frac{2\sin\lt(\frac{3\pi}{4}\rt)}{\frac{3\pi}{4}}+0\rt)\approx 1.836\\
    \int_{0}^{\pi}\frac{\sin x}{x}dx&\approx\frac{\pi}{12}\lt(1+\frac{4\sin\lt(\frac{\pi}{4}\rt)}{\frac{\pi}{4}}+\frac{2\sin\lt(\frac{\pi}{2}\rt)}{\frac{\pi}{2}}+\frac{4\sin\lt(\frac{3\pi}{4}\rt)}{\frac{3\pi}{4}}+0\rt)\approx 1.852
\end{align}

\section{Use the error formulas in Theorem 4.20 (Errors in the Trapezoidal Rule and Simpson's Rule) to
estimate the errors in approximating the integral, with $n=4$
using (a) the Trapezoidal Rule and (b) Simpson’s Rule.}
24.\begin{align}
    \set
    &\int_{3}^{5}(5x+2)dx\\
    f(x)&=5x+2\\
    f'(x)&=5\\
    f''(x)&0
\end{align}
For both rules, the error is zero.

\section{Use the error formulas in Theorem 4.20 to
find $n$ such that the error in the approximation of the definite
integral is less than or equal to 0.00001 using (a) the Trapezoidal
Rule and (b) Simpson’s Rule.}
30.\begin{align}
    \set
    &\int_{0}^{1}\frac{1}{1+x}dx\\
    f(x)&=(1+x)^{-1},\,\, 0\leq x\leq 1\\
    f'(x)&=-(1+x)^{-2}\\
    f''(x)&=2(1+x)^{-3}\\
    f'''(x)&=-6(1+x)^{-4}\\
    f^{(4)}(x)&=24(1+x)^{-5}
\end{align}
\begin{enumerate}[(a)]
    \item Trapezoidal: Maximum of $|f''(x)|=|2(1+x)^{-3}|$ is 2.    
        \begin{align}
            \text{Error}\leq\frac{1}{12n^2}(2)&\leq 0.00001\\
            n^2&\geq16666.67\\
            n&\geq 129.10
        \end{align}
    \item Simpson's: Maximum of $|f^{(4)}(x)|=|24(1+x)^{-5}|$ is 24.
        \begin{align}
            \text{Error}\leq\frac{1}{180n^4}(24)&\leq 0.00001\\
            n^4&\geq133333.33\\
            n&\geq 10.75  
        \end{align}
\end{enumerate}

\vs\next
40. Approximate the area of the shaded region using (a) the
Trapezoidal Rule and (b) Simpson’s Rule with $n=8$\\
\im{40.png}\\
\begin{align}
    \set
    n=8,\,\, b-a&=8-0=8\\
    \int_{0}^{8}f(x)dx&\approx\frac{8}{16}(0+2(1.5)+2(3)+2(5.5)+2(9)+2(10)+2(9)+2(6)+0)=\frac{1}{2}(88)=44\\
    \int_{0}^{8}f(x)dx&\approx\frac{8}{24}(0+4(1.5)+2(3)+4(5.5)+2(9)+4(10)+2(9)+4(6)+0)=\frac{134}{3}
\end{align}

\section{Capstone}
46. Consider a function $f(x)$ that is concave upward on the interval [0, 2] and a function $g(x)$ that is concave downward on [0, 2].
\begin{enumerate}[(a)]
    \item Using the Trapezoidal Rule, which integral would be
    overestimated? Which integral would be underestimated?
    Assume $n=4$ Use graphs to explain your answer.\\
        \indent $\int_{0}^{2}f(x)dx$ would be overestimated, and $\int_{0}^{2}g(x)dx$ would be underestimated.
    \item Which rule would you use for more accurate approxi-
    mations of $\int_{0}^{2}f(x)dx$ and $\int_{0}^{2}g(x)dx$ the Trapezoidal
    Rule or Simpson’s Rule? Explain your reasoning.\\
        \indent Simpson's Rule would be more accurate because it considers the graph's curvature.
\end{enumerate}

\section{Work}
48. To determine the size of the motor required to operate
a press, a company must know the amount of work done when
the press moves an object linearly 5 feet. The variable force to
move the object is $F(x)=100x\sqrt[]{25-x^3}$ where $F$ is given in
pounds and $x$ gives the position of the unit in feet. Use
Simpson’s Rule with $n=12$ to approximate the work $W$ (in
foot-pounds) done through one cycle if $W=\int_{0}^{5}F(x)dx$.
\begin{align}
    \set
    W&=\int_{0}^{5}100x\sqrt[]{125-x^3}dx,\,\, n=12\\
    \int_{0}^{5}100x\sqrt[]{125-x^3}dx&\approx\frac{5}{3(12)}\Bigg(0+400\Bigg(\frac{5}{12}\Bigg)\sqrt[]{125-\Bigg(\frac{5}{12}\Bigg)^3}+200\Bigg(\frac{10}{12}\Bigg)\sqrt[]{125-\Bigg(\frac{10}{12}\Bigg)^3}\Bigg)\nonumber \\
        &+400\Bigg(\frac{15}{12}\sqrt[]{125-\Bigg(\frac{15}{12}\Bigg)^3}+\cdots+0\Bigg)\approx 10233.58\text{ft-lb}
\end{align}

\section{Use Simpson’s
Rule with $n=6$ to approximate $\pi$  using the given equation.}
50.\begin{align}
    \set
    &\int_{0}^{\frac{1}{2}}\frac{6}{\sqrt[]{1-x^2}}dx,\,\, n=6\\
    \pi&\approx\frac{\lt(\frac{1}{2}-0\rt)}{3(6)}(6+4(6.0209)+2(6.0851)+4(6.1968)+2(6.3640)+4(6.6002)+6.9282)\approx\frac{1}{36}(113.098)\approx 3.1416
\end{align}

\section{Use the Trapezoidal Rule to
estimate the number of square meters of land in a lot where $x$
and $y$ are measured in meters, as shown in the figures. The land
is bounded by a stream and two straight roads that meet at right
angles.}
52.\\\im[scale=0.5]{52a.png}\im[scale=0.5]{52b.png}
\begin{align}
    \set
    \text{Area}\approx\frac{1000}{2(10)}(125+2(125)+2(120)+2(112)+2(90)+2(90)+2(95)+2(88)+2(75)+2(35))=89.250\text{m}^2
\end{align}

\section{Warm-up 12/11/2023}
The sides of the rectangle increase in such a way that $\frac{dz}{dt}=1$ and $\frac{dx}{dt}=3\frac{dy}{dt}$. At the instant that $x=4$ and $y=3$ what is the value of $\frac{dx}{dt}$?
\begin{align}
    \set
    x^2+y^2&=z^2\\
    4(3\frac{dy}{dt})+3(\frac{dy}{dt})&=5(1)\\
    \frac{dx}{dt}&=3\frac{dy}{dt}\therefore 15\frac{dy}{dt}=\frac{5}{5\cdot 3}=\frac{1}{3}\frac{dy}{dt}
\end{align}

\vs\next
A polynomial $\p(x)$ has local maxima (-2, 4) and (5, 7) and a local minimum at (1, 1) and no other critical points. How many roots does $p(x)$ have?\\
\indent 2










































































































\end{document}