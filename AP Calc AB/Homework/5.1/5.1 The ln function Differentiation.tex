\documentclass[11pt]{article}
\usepackage{amsmath, amssymb, amsfonts,  graphicx, enumerate, float, wrapfig, hyperref}
\usepackage[margin=0.5in]{geometry}
\graphicspath{{./}}
\newcommand*{\vs}{\vspace{1cm}}
\newcommand*{\next}{\noindent}
\newcommand*{\set}{\setcounter{equation}{0}}
\newcommand*{\im}{\includegraphics}
\newcommand*{\lt}{\left}
\newcommand*{\rt}{\right}

\begin{document}

\title{5.1 The Natural Logarithmic Function: Differentiation}
\author{Juan J. Moreno Santos}
\date{January 2024}

\maketitle
\section{Use a graphing utility to evaluate the
logarithm by (a) using the natural logarithm key and (b) using
the integration capabilities to evaluate the integral $\int_{1}^{x}(\frac{1}{t})dt$}
\begin{enumerate}[(a)]
    \item
        \begin{align}
            \set
            \ln 8.3\approx 2.1163\\
            \int_{1}^{8.3}\frac{1}{t}dt\approx 2.1163
        \end{align}
    \item
        \begin{align}
            \set
            \ln 0.6\approx -0.5108\\
            \int_{1}^{0.6}\frac{1}{t}dt\approx -0.5108
        \end{align}
\end{enumerate}

\section{Match the function with its graph.}
\im{8.png}
8.\begin{align}
    \set
    f(x)=-\ln x
\end{align}
Matches (d) since the graph reflects the x-axis

\vs\next
10.\begin{align}
    \set
    f(x)=-\ln(-x)
\end{align}
Matches (c) since the graph reflects both axes.

\section{Sketch the graph of the function and state its domain.}
12.$f(x)=-2\ln x$\\
Domain: $x>0$\\
\im{12.png}\\

\vs\next
16.$g(x)=2+\ln x$\\
Domain: $x>0$\\
\im{16.png}

\section{In Exercises 19 and 20, use the properties of logarithms to
approximate the indicated logarithms, given that $\ln 2\approx 0.6931$ and $\ln 3\approx 1.0986$.}
20.\begin{enumerate}[(a)]
    \item
        \begin{align}
            \set
            &\ln 0.25\\
            &=\ln\frac{1}{4}\\
            &=\ln 1-\ln 2\ln 2\approx -1.3682
        \end{align}
    \item
        \begin{align}
            \set
            &\ln 24\\
            &=3\ln 2+\ln 3\approx 3.1779
        \end{align}
    \item
        \begin{align}
            \set
            &\ln\sqrt[3]{12}\\
            &=\frac{1}{3}(2\ln 2+\ln 3)\approx 0.8283
        \end{align}
    \item
        \begin{align}
            \set
            &\ln\frac{1}{72}\\
            &=\ln 1-(3\ln 2+2\ln 3)\approx -4.2765
        \end{align}
\end{enumerate}

\section{Use the properties of logarithms to expand
the logarithmic expression.}
22.\begin{align}
    \set
    &\ln\sqrt[]{x^5}\\
    &=\ln x^{\frac{5}{2}}=\frac{5}{2}\ln x
\end{align}

\vs\next
26.\begin{align}
    \set
    &\ln\sqrt[]{a-1}\\
    &=\ln(a-1)^{\frac{1}{2}}=\lt(\frac{1}{2}\rt)\ln(a-1)
\end{align}

\vs\next
28.\begin{align}
    \set
    &\ln 3e^2\\
    &=\ln 3+2\ln e=2+\ln 3
\end{align}

\section{Write the expression as a logarithm of a
single quantity.}
32.\begin{align}
    \set
    &3\ln x+2\ln y-4\ln z\\
    &=\ln x^3+\ln y^2-\ln z^4\\
    &=\ln\frac{x^3y^2}{z^4}
\end{align}

\vs\next
34.\begin{align}
    \set
    &2(\ln x-\ln(x+1)-\ln(x-1))\\
    &=2\ln\frac{x}{(x+1)(x-1)}\\
    &=\ln\lt(\frac{x}{x^2-1}\rt)^2
\end{align}

\section{(a) Verify that $f=g$ by using a graphing utility to graph $f$ and $g$ in the same viewing window and (b) verify that $f=g$ algebraically.}
38.\begin{align}
    \set
    f(x)&=\ln\sqrt[]{x(x^2+1)}\\
    &=\frac{1}{2}\ln(x(x^2+1))\\
    &=\frac{1}{2}(\ln x+\ln(x^2+1))=g(x)
\end{align}

\section{Find the limit.}
42.\begin{align}
    \set
    &\lim_{x\to 5^+}\ln\frac{x}{\sqrt[]{x-4}}\\
    &=\ln 5\approx 1.61
\end{align}

\section{Find an equation of the tangent line to the
graph of the logarithmic function at the point (1, 0).}
44.\begin{align}
    \set
    y&=\ln x^{\frac{3}{2}}=\frac{3}{2}\ln x\\
    y'&=\frac{3}{2x}
\end{align}
The slope at (1, 0) is $\frac{3}{2}$, and the tangent line is:
\begin{align}
    \set
    y-0&=\frac{3}{2}(x-1)\\
    y&=\frac{3}{2}x-\frac{3}{2}
\end{align}

\section{Find the derivative of the function.}
48.\begin{align}
    \set
    f(x)&=\ln(x-1)\\
    f'(x)&=\frac{1}{x-1}
\end{align}

\vs\next
52.\begin{align}
    \set
    y&=x^2\ln x\\
    y'&=x^2\lt(\frac{1}{x}\rt)+2x\ln x\\
    &=x+2x\ln x\\
    &=x(1+2\ln x)
\end{align}

\vs\next
56.\begin{align}
    \set
    y&=\ln(t(t^2+3)^3)\\
    &=\ln t+3\ln(t^2+3)\\
    y'&=\frac{1}{t}+\frac{2}{t^2+3}(2t)\\
    &=\frac{1}{t}+\frac{6t}{t^2+3}
\end{align}

\vs\next
60.\begin{align}
    \set
    h(t)&=\frac{\ln t}{t}\\
    h'(t)&=\frac{t\lt(\frac{1}{t}\rt)-\ln t}{t^2}\\
    &=\frac{1-\ln t}{t^2}
\end{align}

\vs\next
64.\begin{align}
    \set
    y&=\ln\sqrt[3]{\frac{x-1}{x+1}}\\
    &=\frac{1}{3}(\ln(x-1)-\ln(x+1))\\
    y'&=\frac{1}{3}\lt(\frac{1}{x-1}-\frac{1}{x+1}\rt)\\
    &=\frac{1}{3}\cdot\frac{2}{x^2-1}\\
    &=\frac{2}{3(x^2-1)}
\end{align}

\vs\next
68.\begin{align}
    \set
    y&=\frac{-\sqrt[]{x^2+4}}{x}-\frac{1}{4}\ln\lt(\frac{2+\sqrt[]{x^2+4}}{x}\rt)\\
    &=\frac{-\sqrt[]{x^2+4}}{2x^2}-\frac{1}{4}\ln(2+\sqrt[root]{x^2+4})+\frac{1}{4}\ln x\\
    \frac{dy}{dx}&=\frac{-2x^2(\frac{x}{\sqrt[]{x^2+4}})+4x\sqrt[]{x^2+4}}{4x^4}-\frac{1}{4}\lt(\frac{1}{2+\sqrt[]{x^2+4}}\rt)\lt(\frac{x}{\sqrt[]{x^2+4}}\rt)+\frac{1}{4}\ln x\\
    &=\frac{-1}{2x\sqrt[]{x^2+4}}+\frac{\sqrt[]{x^2+4}}{x^3}-\frac{1}{4}\cdot\frac{(2-\sqrt[]{x^2+4})}{-x^2}\lt(\frac{x}{\sqrt[]{x^2+4}}\rt)+\frac{1}{4x}\\
    &=\frac{-1+\lt(\frac{1}{2}\rt)(2-\sqrt[]{x^2+4})}{2x\sqrt[]{x^2+4}}+\frac{\sqrt[]{x^2+4}}{x^3}+\frac{1}{4x}\\
    &=\frac{-\sqrt[]{x^2+4}}{4x\sqrt[]{x^2+4}}+\frac{\sqrt[]{x^2+4}}{x^3}+\frac{1}{4x}\\
    &=\frac{\sqrt[]{x^2+4}}{x^3}
\end{align}

\vs\next
72.\begin{align}
    \set
    y&=\ln|\sec x+\tan x|\\
    \frac{dy}{dx}&=\frac{\sec x\tan+\sec^2x}{\sec x+\tan x}\\
    &=\frac{\sec x(\sec x+\tan x)}{\sec x+\tan x}\\
    &=\sec x
\end{align}

\vs\next
76.\begin{align}
    \set
    g(x)&=\int_{1}^{\ln x}(t^2+3)dt\\
    g'(x)&=((\ln x)^2+3)\frac{d}{dx}(\ln x)\\
    &=\frac{(\ln x)^2+3}{x}
\end{align}

\section{Find an equation of the tangent line to
the graph of at the given point.}
78.\begin{align}
    \set
    f(x)&=3x^2-\ln x,\,\, (0, 4)\\
    \frac{dy}{dx}&=-2x-\frac{1}{\lt(\frac{1}{2}\rt)x+1}\lt(\frac{1}{2}\rt)\\
    &=-2x-\frac{1}{x+2}\\
    \frac{dy}{dx}&=-\frac{1}{2}\,\,\text{when}\,\, x=0\\
    y-4&=-\frac{1}{2}(x-0)\\
    y&=-\frac{1}{2}x+4
\end{align}

\vs\next
82.\begin{align}
    \set
    f(x)&=\frac{1}{2}x\ln(x^2),\,\, (-1, 0)\\
    f'(x)&=\frac{1}{2}\ln(x^2)+\frac{1}{2}\lt(\frac{2x}{x^2}\rt)\\
    &=\frac{1}{2}\ln(x^2)+1\\
    f'(-1)&=1\\
    y-0&=1(x+1)\\
    y&=x+1
\end{align}

\section{Use implicit differentiation to find $\frac{dy}{dx}$}
84.\begin{align}
    \set
    \ln(xy)+5x&=30\\
    \ln x+\ln y+5x&=30\\
    \frac{1}{x}+\frac{1}{y}\cdot\frac{dy}{dx}+5&=0\\
    \frac{1}{y}\cdot\frac{dy}{dx}&=\frac{1}{x}-5\\
    \frac{dy}{dx}&=-\frac{y}{x}-5y\\
    &=-\lt(\frac{y+5xy}{x}\rt)
\end{align}

86.\begin{align}
    \set
    4xy+\ln x^2y&=7\\
    4xy+2\ln x+\ln y&=7\\
    4xy'+4y+\frac{2}{x}+\frac{1}{y}y'&=0\\
    \lt(4x+\frac{1}{y}\rt)y'&=-4y-\frac{2}{x}\\
    y'&=\frac{-4y-\frac{2}{x}}{4x+\frac{1}{y}}\\
    &=\frac{-4xy^2-2y}{4x^2y+x}
\end{align}

\section{Locate any relative extrema and inflection
points.}
92.\begin{align}
    \set
    y&=x-\ln x\\
    \text{Domain:}\,\,&x>0\\
    y'&=1-\frac{1}{x}=0\,\,\text{when}\,\, x=1\\
    y''&=\frac{1}{x^2}>0
\end{align}
Relative minimum at (1, 1).\\

\vs\next
96.\begin{align}
    \set
    y&=x^2\ln\frac{x}{4},\,\,\text{Domain:}\,\, x>0\\
    y'&=x^2\lt(\frac{1}{x}\rt)+2x\ln\frac{x}{4}=x\lt(1+2\ln\frac{x}{4}\rt)=0\,\,\text{when:}\\
    -1&=2\ln\frac{x}{4}\Rightarrow\ln\frac{x}{4}=-\frac{1}{2}\Rightarrow x=4e^{-\frac{1}{2}}\\
    y''&=1+2\ln\frac{x}{4}+2x\lt(\frac{1}{x}\rt)=3+2\ln\frac{x}{4}\\
    y''&=0\,\,\text{when}\,\, x=4e^{-\frac{3}{2}}
\end{align}
Relative minimum at $(4e^{-\frac{1}{2}},\,\,-8e^{-1})$, and point of inflection at $(4e^{-\frac{3}{2}},\,\,-24e^{-3})$.

\section{Use logarithmic differentiation to find $\frac{dy}{dx}$.}
102.\begin{align}
    \set
    y&=\sqrt[]{x^2(x+1)(x+2)},\,\, x>0\\
    y^2&=x^2(x+1)(x+2)\\
    2\ln y&=2\ln x+\ln(x+1)+\ln(x+2)\\
    \frac{2}{y}\cdot\frac{dy}{dx}&=\frac{2}{x}+\frac{1}{x+1}+\frac{1}{x+2}\\
    \frac{dy}{dx}&=\frac{y}{2}\lt(\frac{2}{x}+\frac{1}{x+1}+\frac{1}{x+2}\rt)\\
    \frac{dy}{dx}&=\frac{\sqrt[]{x^2(x+1)(x+2)}}{2}\lt(\frac{2(x+1)(x+2)+x(x+2)+x(x+1)}{x(x+1)(x+2)}\rt)\\
    &=\frac{4x^2+9x+4}{2\sqrt[]{(x+1)(x+2)}}
\end{align}

\section{Determine whether the
statement is true or false. If it is false, explain why or give an
example that shows it is false.}
111. $\ln(x+25)=\ln x+\ln 25$\\
\indent False because $\ln x+\ln 25=\ln(25x)\neq\ln(x+25)$.

\vs\next
112. $\ln xy=\ln x\ln y$\\
\indent False because the property actually is $\ln xy=\ln x+\ln y$.

\vs\next
113. If $y=\ln\pi$, then $y'=\frac{1}{\pi}$.\\
\indent False. Since $\pi$ is a constant, $\frac{d}{dx}(\ln\pi)=0$

\vs\next
114. If $y=\ln e$, then $y'=1$.\\
\indent False because if $y=\ln e=1$, then $y'=0$.

\section{Word problems}
116. The relationship between the number of
decibels $\beta$ and the intensity of a sound $I$ in watts per
centimeter squared is $\beta=10\log_{10}\lt(\frac{1}{10^{-16}}\rt)$.\\
Use the properties of logarithms to write the formula in
simpler form, and determine the number of decibels of a
sound with an intensity of $10^{-10}$ watt per square centimeter.\\
\begin{align}
    \set
    \beta&=10\log_{10}\lt(\frac{I}{10^{-16}}\rt)\\
    &=\frac{10}{\ln 10}(\ln I+16\ln 10)\\
    &=160+10\log_{10}I\\
    B(10^{-10})&=\frac{10}{\ln 10}(\ln 10^{-10}+16\ln 10)\\
    &=\frac{10}{\ln 10}(-10\ln 10+16\ln 10)\\
    &=\frac{10}{\ln 10}(6\ln 10)\\
    &=60\,\,\text{decibels}
\end{align}




























\end{document}