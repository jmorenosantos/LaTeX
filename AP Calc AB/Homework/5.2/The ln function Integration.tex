\documentclass[11pt]{article}
\usepackage{amsmath, amssymb, amsfonts, gensymb,  graphicx, enumerate, float, wrapfig, hyperref}
\usepackage[margin=0.5in]{geometry}
\graphicspath{{./}}
\newcommand*{\vs}{\vspace{1cm}}
\newcommand*{\next}{\noindent}
\newcommand*{\set}{\setcounter{equation}{0}}
\newcommand*{\im}{\includegraphics}
\newcommand*{\lt}{\left}
\newcommand*{\rt}{\right}

\begin{document}

\title{5.2 The Natural Logarithmic Function: Integration}
\author{Juan J. Moreno Santos}
\date{January 2024}

\maketitle
\section{Find the indefinite integral}
2.\begin{align}
    \set
    &\int\frac{10}{x}dx\\
    &=10\int\frac{1}{x}dx\\
    &=10\ln|x|+C
\end{align}

\vs\next
6.\begin{align}
    \set
    &\int\frac{1}{4-3x}dx,\,\, u=4-3x,\,\, du=-3dx\\
    &=-\frac{1}{3}\int\frac{1}{4-3x}(-3)dx\\
    &=-\frac{1}{3}\ln|4-3x|+C
\end{align}

\vs\next
10.\begin{align}
    \set
    &\int\frac{x^2-2x}{x^3-3x^2}dx,\,\, u=x^3-3x^2,\,\, du=(3x^2-6x)dx=3(x^2-2x)dx\\
    &=\frac{1}{3}\int\frac{1}{x^3-3x^2}(3x^2-6x)dx\\
    &=\frac{1}{3}\ln|x^3-3x^2|+C
\end{align}

\vs\next
14.\begin{align}
    \set
    &\int\frac{x(x+2)}{x^3+3x^2-4}dx,\,\, u=x^3+3x^2-4,\,\, du=(3x^2+6x)dx\\
    &=\frac{1}{3}\int\frac{3x^2+6x}{x^3+3x^2-4}dx\\
    &=\frac{1}{3}\ln|x^3+3x^2-4|+C
\end{align}

\vs\next
18.\begin{align}
    \set
    &\int\frac{x^3-6x-20}{x+5}dx\\
    &=\int\lt(x^2-5x+19-\frac{115}{x+5}\rt)dx\\
    &=\frac{x^3}{3}-\frac{5x^2}{2}+19x-115\ln|x+5|+C
\end{align}

\vs\next
22.\begin{align}
    \set
    &\int\frac{1}{x\ln(x^3)}dx\\
    &=\frac{1}{3}\int\frac{1}{\ln x}\cdot\frac{1}{x}dx\\
    &=\frac{1}{3}\ln|\ln|x||+C
\end{align}

\vs\next
26.\begin{align}
    \set
    &\int\frac{x(x-2)}{(x-1)^3}dx\\
    &=\int\frac{x^2-2x+1-1}{(x-1)^3}dx\\
    &=\int\frac{(x-1)^2}{(x-1)^3}dx-\int\frac{1}{(x-1)^3}dx\\
    &=\int\frac{1}{x-1}dx-\int\frac{1}{(x-1)^3}dx\\
    &=\ln|x-1|+\frac{1}{2(x-1)^2}+C
\end{align}

\vs\next
28.\begin{align}
    \set
    &\int\frac{1}{1+\sqrt[]{3x}}dx,\,\, u=1+\sqrt[]{3x},\,\, du=\frac{3}{2\sqrt[]{3x}}dx\Rightarrow dx=\frac{2}{3}(u-1)du\\
    &=\int\frac{1}{u}\cdot\frac{2}{3}(u-1)du\\
    &=\frac{2}{3}\int\lt(1-\frac{1}{u}\rt)du\\
    &=\frac{2}{3}(u-\ln|u|)+C\\
    &=\frac{2}{3}(1+\sqrt[]{3x}-\ln(1+\sqrt[]{3x}))+C\\
    &=\frac{2}{3}\sqrt[]{3x}-\frac{2}{3}\ln(1+\sqrt[]{3x})+C
\end{align}

\vs\next
30.\begin{align}
    \set
    &\int\frac{\sqrt[3]{x}}{\sqrt[3]{x}-1}dx,\,\, u=x^{\frac{1}{3}}-1,\,\, du=\frac{1}{3x^{\frac{2}{3}}}dx\Rightarrow dx=3(u+1)^2du\\
    &=\int\frac{u+1}{u}3(u+1)^2du\\
    &=3\int\frac{u+1}{u}(u^2+2u+1)du\\
    &=3\int\lt(u^2+3u+3=\frac{1}{u}\rt)du\\
    &=3\lt(\frac{u^3}{3}+\frac{3u^2}{2}+3u+\ln|u|\rt)+C\\
    &=3\lt(\frac{(x^{\frac{1}{3}}-1)^3}{3}+\frac{3(x^{\frac{1}{3}}-1^2)}{2}+3(x^{\frac{1}{3}}-1)+\ln|x^{\frac{1}{3}}-1|\rt)+C\\
    &=3\ln|x^{\frac{1}{3}}-1|+\frac{3x^{\frac{2}{3}}}{2}+3x^{\frac{1}{3}}+x+C
\end{align}

\vs\next
32.\begin{align}
    \set
    &\int\tan 5\theta d\theta\\
    &=\frac{1}{5}\int\frac{5\sin 5\theta}{\cos 5\theta}d\theta\\
    &=-\frac{1}{5}\ln|\cos 5\theta|+C
\end{align}

\vs\next
36.\begin{align}
    \set
    &\int\lt(2-\tan\frac{\theta}{4}\rt)d\theta\\
    &=\int 2d\theta-4\int\tan\frac{\theta}{4}\lt(\frac{1}{4}\rt)d\theta\\
    &=2\theta+4\ln\lt|\cos\frac{\theta}{4}\rt|+C
\end{align}

\vs\next
40.\begin{align}
    \set
    &\int(\sec 2x=\tan 2x)dx\\
    &=\frac{1}{2}\int(\sec 2x+\tan 2x)(2)dx\\
    &=\frac{1}{2}(\ln|\sec 2x+\tan 2x|-\ln|\cos 2x)+C
\end{align}

\vs\next
\section{Solve the differential equation.}
42.\begin{align}
    \set
    \frac{dy}{dx}&=\frac{x-2}{x},\,\,(-1,\, 0)\\
    y&=\int\frac{x-2}{x}dx\\
    &=\int\lt(1-\frac{2}{x}\rt)dx\\
    &=x-2\ln|x|+C\\
    0&=-1-2\ln|-1|+C=-1+C\Rightarrow C=1\\
    y&=x-2\ln|x|+1
\end{align}

46.\begin{align}
    \set
    \frac{dr}{dt}&=\frac{\sec^2t}{\tan t+1}
\end{align}

\section{A differential equation, a
point, and a slope field are given. (a) Sketch two approximate
solutions of the differential equation on the slope field, one of
which passes through the given point. (b) Use integration to find
the particular solution of the differential equation and use a
graphing utility to graph the solution. Compare the result with
the sketches in part (a). }
49.\begin{align}
    \set
    \frac{dy}{dx}=\frac{1}{x+2},\,\, (0,\, 1)
\end{align}
\begin{enumerate}[(a)]
    \item \im{50.png}\\
    \item
        \begin{align}
            \set
            y&=\int\frac{\ln x}{x}dx=\frac{(\ln x)^2}{2}+C\\
            y(1)&=-2\Rightarrow -2=\frac{(\ln 1)^2}{2}+C\therefore C=-2\\
            y&=\frac{(\ln x)^2}{2}-2
        \end{align}
\end{enumerate}

\section{Evaluate the definite integral.}
58.\begin{align}
    \set
    &\int_{0}^{1}\frac{x-1}{x+1}dx\\
    &=\int_{0}^{1}1dx+\int_{0}^{1}\frac{-2}{x+1}dx\\
    &=(x-2\ln|x+1|)_0^1=1-2\ln 2\\
    &\approx -0.386
\end{align}

\section{Find $F'(x)$.}
68.\begin{align}
    \set
    F(x)&=\int_{0}^{x}\tan tdt\\
    F'(x)&=\tan x
\end{align}

\vs\next
70.\begin{align}
    \set
    F(x)&=\int_{1}^{x^2}\frac{1}{t}dt\\
    F'(x)&=\frac{2x}{x^2}=\frac{2}{x}
\end{align}

72.\section{Determine which value
best approximates the area of the region between the $x$-axis and
the graph of the function over the given interval. (Make your
selection on the basis of a sketch of the region and not by
performing any calculations.)}
72.$f(x)=\frac{2x}{x^2+1},\,\, [0,\, 4]$
\begin{enumerate}[(a)]
    \item 3
    \item 7
    \item 2
    \item 5
    \item 1
\end{enumerate}
\im{72.png}\\
$A\approx 3\therefore$ it matches (a).

\section{Find the area of the given region.}
74.\begin{align}
    \set
    y&=\frac{2}{x\ln x}\\
    A&=\int_{2}^{4}\frac{2}{x\ln x}\\
    &=2\int_{2}^{4}\frac{1}{\ln x}\cdot\frac{1}{x}dx\\
    &=2\ln|\ln x|)^4_2\\
    &=2(\ln(\ln 4)-\ln(\ln 2))\\
    &=2\ln\lt(\frac{2\ln 2}{\ln 2}\rt)\\
    &=2\ln 2
\end{align}
\im{74.png}

\vs\next
76.\begin{align}
    \set
    y&=\frac{\sin x}{1+\cos x}\\
    A&=\frac{\frac{\pi}{4}}{\frac{3\pi}{4}}\frac{\sin x}{1+\cos x}dx\\
    &=-\ln|1+\cos x|)_{\frac{\pi}{4}}^{\frac{3\pi}{4}}\\
    &=-\ln\lt(1-\frac{\sqrt[]{2}}{2}\rt)+\ln\lt(1+\frac{\sqrt[]{2}}{2}\rt)\\
    &=\ln\lt(\frac{2+\sqrt[]{2}}{2-\sqrt[]{2}}\rt)\\
    &=\ln(3+2\sqrt[]{2} )
\end{align}
\im{76.png}

\section{Find the area of the region bounded
by the graphs of the equations.}
78.\begin{align}
    \set
    y&=\frac{x+6}{x},\,\, x=1,\,\, x=5,\,\, y=0\\
    A&=\int_{1}^{5}\frac{x+6}{x}dx\\
    &=\int_{1}^{5}\lt(1+\frac{6}{x}\rt)dx\\
    &=(x+6\ln x)_1^5\\
    &=5+6\ln 5-1\\
    &=4+6\ln 5\\
    &\approx 13.657
\end{align}


\vs\next
80.\begin{align}
    \set
    y&=2x-\tan 0.3x,\,\, x=1,\,\, x=4,\,\, y=0\\
    &\int_{1}^{4}(2x-\tan(0.3x))dx\\
    &=\lt(x^2+\frac{10}{3}\ln|\cos(0.3x)|\rt)_1^4\\
    &=\lt(16+\frac{10}{3}\ln\cos(1.2)\rt)-\lt(1+\frac{10}{3}\ln\cos(0.3)\rt)\\
    &\approx 11.769
\end{align}

\section{Use the Trapezoidal
Rule and Simpson’s Rule to approximate the value of the
definite integral. Let $n=4$ and round your answer to four
decimal places.}
82.\begin{align}
    \set
    f(x)&=\frac{8x}{x^2+4},\,\, b-4=4-0-4,\,\, n=4\\
    \text{Trapezoidal Rule:}&\frac{4}{2(4)}(f(0)+2f(1)+2f(2)+2f(3)+f(4))=\frac{1}{2}(0+3.2+4+3.6923+1.6)\approx 6.2462\\
    \text{Simpson's Rule:}&\frac{4}{3(4)}(f(0)+4f(1)+2f(2)+4f(3)+f(4))\approx 6.4615
\end{align}

\section{Capstone}
90. Find a value of $x$ such that $\int_{1}^{x}\frac{1}{t}dt$ is equal to (a) $\ln 5$ and (b) $1$.
\begin{align}
    \set
    \int_{1}^{x}\frac{1}{t}dt&=(\ln|t|)_1^x=\ln x
\end{align}
\begin{enumerate}[(a)]
    \item\begin{align}
        \ln x=\ln 5\Rightarrow x=5
    \end{align}
    \item\begin{align}
        \ln x=1\Rightarrow x=e
    \end{align}
\end{enumerate}

\section{Find the average value of the function over the given interval.}
98.\begin{align}
    \set
    f(x)&=\frac{4(x+1)}{x^2},\,\,[2,\, 4]\\
    \text{Average}&=\frac{1}{4-2}\int_{2}^{4}\frac{4(x+1)}{x^2}dx\\
    &=2\int_{2}^{4}\lt(\frac{1}{x}+\frac{1}{x^2}\rt)dx\\
    &=2\lt(\ln x-\frac{1}{x}\rt)_2^4\\
    &=2\lt(\ln 4-\frac{1}{4}-\ln 2+\frac{1}{2}\rt)\\
    &=2\lt(\ln 2+\frac{1}{4}\rt)\\
    &=\ln 4+\frac{1}{2}\\
    &\approx 1.8863
\end{align}
\im{98.png}

100.\begin{align}
    \set
    f(x)&=\sec\frac{\pi x}{6},\,\, [0,\, 2]\\
    \text{Average}&=\frac{1}{2-0}\int_{0}^{2}\sec\frac{\pi x}{6}dx\\
    &=\lt(\frac{1}{2}\lt(\frac{6}{\pi}\ln\lt|\sec\frac{\pi x}{6}+\tan\frac{\pi x}{6}\rt|\rt)\rt)_0^2\\
    &=\frac{3}{\pi}(\ln(2+\sqrt[]{3})-\ln(1+0))\\
    &=\frac{3}{\pi}\ln(2+\sqrt[]{3})
\end{align}

\section{Word problems}
102. Find the time required for an object to cool from $300\degree F$ to $250\degree F$ by evaluating $t=\frac{10}{\ln 2}\int_{250}^{300}\frac{1}{T-100}dT$ where $t$ is time in minutes.
\begin{align}
    \set
    t&=\frac{10}{\ln 2}\int_{250}^{300}\frac{1}{T-100}dT\\
    &=\frac{10}{\ln 2}(\ln(T-100))_{250}^{300}\\
    &=\frac{10}{\ln 2}(\ln 200-\ln 150)\\
    &=\frac{10}{\ln 2}\lt(\ln\lt(\frac{4}{3}\rt)\rt)\approx 4.1504\text{min}
\end{align}

\vs\next
104. The rate of change in sales $S$ is inversely proportional to time $t(t>1)$ measured in weeks. Find $S$ as a function of $t$ if sales after 2 and 4 weeks are 200 units and 300 units, respectively.
\begin{align}
    \set
    \frac{dS}{dt}&=\frac{k}{t}\\
    S(t)&=\int\frac{k}{t}dt\\
    &=k\ln|t|+C\\
    &=k\ln t+C\because t>1\\
    S(2)&=k\ln 2+C=200\\
    S(4)&=k\ln 4+C=300\therefore k=\frac{100}{\ln 2},\,\, C=100\\
    S(t)&=\frac{100\ln t}{\ln 2}+100\\
    &=100\lt(\frac{\ln t}{\ln 2}+1\rt)
\end{align}

\section{Determine whether the
statement is true or false. If it is false, explain why or give an
example that shows it is false.}
105.$(\ln x)^{1/2}=\frac{1}{2}(\ln x)$\\
\indent False because $\frac{1}{2}(\ln x)=\ln(x^{1/2})\neq(\ln x)^{1/2}$.

\vs\next
106.$\int\ln xdx=\lt(\frac{1}{x}\rt)+C$\\
\indent False because $\frac{d}{dx}(\ln x)=\frac{1}{x}$.

\vs\next
107.$\int\frac{1}{x}dx=\ln|cx|,\,\, c\neq 0$\\
\indent True.

\vs\next
108.$\int_{-1}^{2}\frac{1}{x}dx=(\ln|x|)_{-1}^2=\ln 2-\ln 1=\ln 2$\\
\indent False because $\frac{1}{x}$ is discontinuous at $x=0$







\end{document}