\documentclass[11pt]{article}
\usepackage{amsmath, amssymb, amsfonts, gensymb,  graphicx, enumerate, float, wrapfig, hyperref}
\usepackage[margin=0.5in]{geometry}
\graphicspath{{./}}
\newcommand*{\vs}{\vspace{1cm}}
\newcommand*{\next}{\noindent}
\newcommand*{\set}{\setcounter{equation}{0}}
\newcommand*{\im}{\includegraphics}
\newcommand*{\lt}{\left}
\newcommand*{\rt}{\right}

\begin{document}

\title{5.3 Inverse functions}
\author{Juan J. Moreno Santos}
\date{January 2024}

\maketitle
\section{The graph of the function with the
graph of its inverse function. [The graphs of the inverse
functions are labeled (a), (b), (c), and (d).]}
\im[scale=0.75]{12.png}\\
10 matches (b) and 12 matches (d)

\section{Use a graphing utility to graph the function.
Then use the Horizontal Line Test to determine whether the
function is one-to-one on its entire domain and therefore has an
inverse function.}
16. $f(x)=\frac{x^2}{x^2+4}$\\
\indent The function is not one-to-one and doesn't have an inverse.

\section{Find the inverse function of $f$, (b) graph $f$ and $f^{-1}$ on the same set of coordinate axes, (c) describe the relationship between the graphs, and (d) state the domain and range of $f$ and $f^{-1}$.}
28.\begin{align}
    \set
    f(x)=x^2&=y,\,\, x\geq 0\\
    x&=\sqrt[]{y}\\
    y&=\sqrt[]{y}\\
    f^{-1}(x)&=\sqrt[]{x}
\end{align}
$f$ and $f^{-1}$ reflect each other on the line $y=x$. The domain and range of $f$ are $x\geq 0$ and $y\geq 0$ respectively, and the domain and range of $f^{-1}$ are $x\geq 0$ and $y\geq 0$ respectively.

\section{Word problems}
40. The formula $C=\frac{5}{9}(F-32)$, where $F\geq 459.6$, represents Celsius temperature $C$ as a function of Fahrenheit temperature $F$.
\begin{enumerate}[(a)]
    \item Find the inverse function of C
        \begin{align}
            \set
            \frac{9}{5}&F-32\\
            F=32+\frac{9}{5}C
        \end{align}
    \item What does the inverse function represent?
        \indent It represents the temperature conversion from degrees Celsius to Fahrenheit.
    \item What is the domain of the inverse function> Validate or explain your answer using the context of the problem.
        \begin{align}
            \set
            F\geq -459.6,\,\, C=\frac{5}{9}(F-32)\geq -273.11
            \therefore\text{Domain:} C\geq -273.1=-273\frac{1}{9}
        \end{align}
    \item The temperature is 22\degree C. What is the corresponding temperature in degrees Fahrenheit?
        \begin{align}
            \set
            F=32+\frac{9}{5}(22)=71.6\degree F
        \end{align}
\end{enumerate}

\section{Use the derivative to determine whether the
function is strictly monotonic on its entire domain and therefore
has an inverse function.}
42.\begin{align}
    \set
    f(x)&=x^3-6x^2+12x\\
    f'(x)&=3x^2-12x+12=3(x-2)^2
\end{align}
$f$ is strictly monotonic and has an inverse because it's increasing on $(-\infty,\,\infty)$.

46.\begin{align}
    \set
    f(x)&=\cos\frac{3x}{2}\\
    f'(x)&=-\frac{3}{2}\sin\frac{3x}{2}=0\,\,\text{when}\,\, x=0,\,\frac{2\pi}{3},\,\frac{4\pi}{3},\cdots
\end{align}

\section{Determine whether the functions is one-to-one. If it is, find its inverse function.}
60. $f(x)=-3$\\
\indent Doesn't have an inverse since it's not one-to-one.

\vs\next
62.\begin{align}
    \set
    f(x)&=ax+b
\end{align}
$f$ is one-to-one and has an inverse.
\begin{align}
    \set
    ax+b&=y\\
    x&=\frac{y-b}{a}\\
    y&=\frac{x-b}{a}
\end{align}

\section{Delete part of the domain so that the
function that remains is one-to-one. Find the inverse function of
the remaining function and give the domain of the inverse
function.}
64.\begin{align}
    \set
    f(x)&=16-x^4\,\,\text{will be one-to-one for}\,\, x\geq 0\\
    16-x^4&=y\\
    16-y&=x^4\\
    \sqrt[4]{16-y}&=x^4\\
    \sqrt[4]{16-x}&=y\\
    f^{-1}(x)&=\sqrt[4]{16-x},\,\, x\leq 16
\end{align}
\im{64.png}

\section{Decide whether the function has an inverse function. If so, what is the inverse function?}
68. $h(t)$ is the height of the tide $t$ hours after midnight, where $0\leq t\leq 24$.\\
\indent The function doesn't have an inverse because there could be two times $t_1\neq t_2$ for which $h(t_1)=h(t_2)$.

70. $A(r)$ is the area of a circle of radius $r$.\\
\indent Yes. The function is one-to-one since it's increasing. Its inverse yields the radius $r$ that corresponds to the area A.

\section{Verify that $f$ has an inverse. Then use the function $f$ and the given real number $a$ to find $(f^{-1})'(a)$.}
72.\begin{align}
    \set
    f(x)&=5-2x^3,\,\, a=7\\
    f'(x)&=-6x^2
\end{align}
$f$ is decreasing on $(-\infty,\,\,\infty)$. Therefore, $f$ has an inverse and is monotonic.\\
\begin{align}
    f(-1)&=7\Rightarrow f^{-1}(7)=-1\\
    (f^{-1})'(7)&=\frac{1}{f'(f^{-1}(7))}=\frac{1}{f'(-1)}=\frac{1}{-6(-1)^2}=\frac{-1}{6}
\end{align}

\vs\next
76.\begin{align}
    \set
    f(x)&=\cos 2x,\,\, a=1,\,0\leq x\leq\frac{\pi}{2}\\
    f'(x)&=-2\sin 2x<0\,\,\text{on}\,\, (0,\,\frac{\pi}{2})
\end{align}
$f$ has an inverse since it's monotonic on $[0,\,\frac{\pi}{2}]$.
\begin{align}
    f(0)&=1\Rightarrow f^{-1}(1)=0\\
    (f^{-1})'(1)&=\frac{1}{f'(f^{-1}(1))}=\frac{1}{f'(0)}=\frac{1}{-2\sin 0}=\frac{1}{0}=\text{undefined}
\end{align}

\vs\next
80.\begin{align}
    \set
    f(x)&=\sqrt[]{x-4},\,\, a=2,\, x\geq 4\\
    f'(x)&=\frac{1}{2\sqrt[]{x-4}}>0\,\,\text{on}\,\, (4,\,\infty)
\end{align}
$f$ has an inverse since it's monotonic on [4, $\infty$]

\section{Find the domains of $f$ and $f^{-1}$, find the ranges of $f$ and $f^{-1}$, graph $f$ and $f^{-1}$, and show that the slopes of the graphs of $f$ and $f^{-1}$ are reciprocals at the given point.}
82.\begin{align}
    \set
    f(x)&=3-4x,\,\, (1,\, -1);\,\, f^{-1}=\frac{3-x}{4},\,\, (-1,\, 1)\\
    \text{Domain}\,\, f&=\text{Domain}\,\, f^{-1}=(-\infty,\,\infty)\\
    \text{Range}\,\, f&=\text{Range}\,\, f^{-1}=(-\infty,\,\infty)\\
    f'(x)&=-4\\
    f'(-1)&=-4\\
    (f^{-1})^{'}(x)&=-\frac{1}{4}\\
    (f^{-1})^{'}(-1)&=-\frac{1}{4}
\end{align}

\section{Find $\frac{dy}{dx}$ for the equation at the given point.}
86.\begin{align}
    \set
    x&=2\ln(y^2-3),\,\, (0,\, 2)\\
    1&=2\frac{1}{y^2-3}2y\frac{dy}{dx}\\
    \frac{dy}{dx}&=\frac{y^2-3}{4y}\\
    &=\frac{4-3}{8}=\frac{1}{8}
\end{align}

\section{Use the functions $f(x)=\frac{1}{8}x-3$ and $g(x)=x^3$ to find the given value.}
88.\begin{align}
    \set
    &(g^{-1}\circ f^{-1})(-3)\\
    &=g^{-1}(f^{-1}(-3))\\
    &=g^{-1}(0)=0
\end{align}

\section{Use the functions $f(x)x+4$ and $g(x)=2x-5$ to find the given function.}
94.\begin{align}
    \set
    (g\circ f)(x)&=g(f(x))\\
    &=g(x+4)\\
    &=2(x+4)-5\\
    &=2x+3\\
    (g\circ f)^{-1}(x)&=\frac{x-3}{2}
\end{align}

\section{Determine whether the
statement is true or false. If it is false, explain why or give an
example that shows it is false.}
101. If $f$ is an even function, then $f^{-1}$ exists.\\
\indent False. This isn't true with $f(x)=x^2$.

\vs\next
102. If the inverse function of $f$ exists, then the y-intercept of $f$ is an $x$-intercept of $f^{-1}$.\\
\indent True.

\vs\next
103. If $f(x)=x^n$, where $n$ is odd, then $f^{-1}$ exists.\\
\indent True.

\vs\next
104. There exists no function $f$ such that $f=f^{-1}$.\\
\indent False. One of these function is $f(x)=x$.


\end{document}