\documentclass[11pt]{article}
\usepackage{amsmath, amssymb, amsfonts, gensymb,  graphicx, enumerate, float, wrapfig, hyperref}
\usepackage[margin=0.5in]{geometry}
\graphicspath{{./}}
\newcommand*{\vs}{\vspace{1cm}}
\newcommand*{\next}{\noindent}
\newcommand*{\set}{\setcounter{equation}{0}}
\newcommand*{\im}{\includegraphics}
\newcommand*{\lt}{\left}
\newcommand*{\rt}{\right}
\newcommand*{\s}{\section}
\newcommand*{\Ra}{\Rightarrow}

\begin{document}

\title{5.4 Exponential Functions: Differentiation and Integration}
\author{Juan J. Moreno Santos}
\date{January 2024}

\maketitle
\s{Solve for $x$ accurate to three decimal places.}
2.\begin{align}
    \set
    e^{\ln 2x}&=12\\
    2x&=12\Ra x=6
\end{align}

10.\begin{align}
    \set
    \frac{5000}{1+e^{2x}}&=2\\
    \frac{5000}{2}&=1+e^{2x}\\
    2499&=e^{2x}\\
    \ln 2499&=2x\\
    x&=\frac{1}{2}\ln 2499\approx 3.912
\end{align}

\s{Sketch the graph of the function.}
20. $y=e^{x-1}$\\
\im{20.png}

\s{Match the equation with the correct graph. Assume that $a$ and $C$ are positive real numbers.}
\im{26.png}\\
26.$y=Ce^{-ax}$\\
Horizontal asymptote at $y=0$\\
Reflects the $y$-axis and matches (d)

28.\begin{align}
    \set
    y&=\frac{C}{1+e^{-ax}}\\
    &\lim_{x\to\infty}\frac{C}{1+e^{-ax}}=C\\
    &\lim_{x\to\infty}\frac{C}{1+e^{-ax}}=0\\
\end{align}

\s{Illustrate that the functions are inverses of
each other by graphing both functions on the same set of
coordinate axes}
30.\begin{align}
    \set
    f(x)&=e^{x/3}\\
    g(x)&=\ln x^3=3\ln x
\end{align}
\im{30.png}

\s{Find an equation of the tangent line to the graph of the function at the point (0, 1)}
38.\begin{align}
    \set
    y&=e^{2x}\\
    y'&=2e^{2x}\\
    y'(0)=2\\
    \text{Tangent line}=y-1&=2(x-0)\\
    y&=2x+1
\end{align}

\begin{align}
    \set
    y&=e{-2x}\\
    y'&=-2e^{-2x}\\
    y'(0)&=-2\\
    \text{Tangent line}=y-1&=-2(x-0)\\
    y&=-2x+1
\end{align}

\s{Find the derivative}
40.\begin{align}
    \set
    y&=e^{-5x}\\
    \frac{dy}{dx}&=-5e^{-5x}
\end{align}

44.\begin{align}
    \set
    f(x)&=3e^{1-x^2}\\
    f'(x)&=3e^{1-x^2}(-2x)\\
    &=-6xe^{1-x^2}
\end{align}

48.\begin{align}
    \set
    y&=x^2e^{-x}\\
    y'&=x^2(-e^{-x})+2xe^{-x}\\
    &=xe^{-x}(2-x)
\end{align}

52.\begin{align}
    \set
    y&=\ln\lt(\frac{1+e^x}{1-e^x}\rt)\\
    &=\ln(1+e^x)-\ln(1-e^x)\\
    \frac{dy}{dx}&=\frac{e^x}{1+e^x}+\frac{e^x}{1-e^x}\\
    &=\frac{2e^x}{1-e^{2x}}
\end{align}

56.\begin{align}
    \set
    y&=\frac{e^{2x}}{e^{2x}+1}\\
    y'&=\frac{(e^{2x}+1)2e^{2x}-e^{2x}(2e^{2x})}{(e^{2x}+1)^2}\\
    &=\frac{2e^{2x}}{(e^{2x}+1)^2}
\end{align}

60.\begin{align}
    \set
    F(x)&=\int_{0}^{e^{2x}}\ln(t+1)dt\\
    F'(x)&\ln(e^{2x}+1)2e^{2x}\\
    &=2e^{2x}\ln(e^{2x}+1)
\end{align}

\s{Find an equation of the tangent line to the graph of the function at the given point.}
62.\begin{align}
    \set
    y&=e^{-2x+x^2},\,\, (2,\, 1)\\
    y'&=(2x-2)e^{-2x+x^2},\,\, y'(2)=2\\
    \text{Tangent line}=y-1&=2(x-2)\\
    y&=2x-3
\end{align}

66.\begin{align}
    \set
    y&=xe^x-e^x,\,\, (1,\, 0)\\
    y'&=xe^x+e^x-e^x=xe^x\\
    y'(1)&=e\\
    \text{Tangent line}=y-0&=e(x-1)\\
    y&=ex-e
\end{align}

\s{Use implicit differentiation to find $\frac{dy}{dx}$.}
\begin{align}
    \set
    e^{xy}+x^2-y^2&=10\\
    \lt(x\frac{dy}{dx}+y\rt)e^{xy}+2x-2y\frac{dy}{dx}&=0\\
    \frac{dy}{dx}(xe^{xy}-2y)&=-ye^{xy}-2x\\
    \frac{dy}{dx}&=-\frac{ye^{xy}+2x}{xe^{xy}-2y}
\end{align}

\s{Find an equation of the tangent line to
the graph of the function at the given point.}
72.\begin{align}
    \set
    1+\ln(xy)&=e^{x-y},\,\, (1,\, 1)\\
    \frac{1}{xy}(xy'+y)&=e^{x-y}(1-y')\\
    y'+1&=1-y'\\
    y'&=0
\end{align}

\s{Find the second derivative of the function.}
74.\begin{align}
    \set
    g(x)&=\sqrt[]{x}+e^x\ln x\\
    g'(x)&=\frac{1}{2\sqrt[]{x}}+\frac{e^x}{x}+e^x\ln x\\
    g''(x)&=-\frac{1}{4x^{3/2}}+\frac{xe^x-e^x}{x^2}+\frac{e^x}{x}+e^x\ln x\\
    &=-\frac{1}{4x\sqrt[]{x}}+\frac{e^x(2x-1)}{x^2}+e^x\ln x
\end{align}

\s{Find the extrema and the points of inflection
(if any exist) of the function}
80.\begin{align}
    \set
    f(x)&=\frac{e^x-e^{-x}}{2}\\
    f'(x)&=\frac{e^x+e^{-x}}{2}>0\\
    f''(x)&=\frac{e^x-e^{-x}}{2}=0\,\,\text{when}\,\, x=0\\
    &\text{Inflection point at (0, 0)}
\end{align}

84.\begin{align}
    \set
    f(x)&=xe^{-x}\\
    f'(x)&=-xe^{-x}+e^{-x}\\
    &=e^{-x}(1-x)=0\,\,\text{when}\,\, x=1\\
    f''(x)&=-e^{-x}+(-e^{-x})(1-x)\\
    &=e^{-x}(x-2)=0\,\,\text{when}\,\, x=2\\
    &\text{Relative maximum at }(1, e^{-1})\\
    &\text{Inflection point at }(2, 2e^{-2})
\end{align}

\s{Word problems}
90. Find the point on the graph of $y=e^{-x}$ where the normal line to
the curve passes through the origin.\\
\indent Let $(x_0,\, y_0)$ be the coordinate on the graph.
\begin{align}
    \set
    y&=e^{-x}\\
    y'&=-e^{-x}\\
    -\frac{1}{y'}&=e^x\\
    y-e^{-x_0}&=e^{x_0}(x-x_0)\\
\end{align}
Because the curve passes through the origin,
\begin{align}
    -e^{-x_0}&=-x_0e^{x_0}\\
    1&=x_0e^{2x_0}\\
    x_0e^{2x_0}-1&=0\therefore x_0\approx 0.4263
\end{align}

\vs\next
92. The displacement from equilibrium of a
mass oscillating on the end of a spring suspended from a
ceiling is $y=1.56e^{-0.22t}\cos 4.9t$, where $y$ is the displacement
in feet and $t$ is the time in seconds. Use a graphing utility to
graph the displacement function on the interval [0, 10] Find a
value of $t$ past which the displacement is less than 3 inches from
equilibrium.\\
\indent $1.56e^{-0.22t}\cos 4.9t\leq 0.25$. With a calculator, it is determined that $t\geq 7.79$ seconds.

\vs\next
94. The table lists the approximate values $V$ of a mid-sized sedan for the years 2003 through 2009. The variable $t$ represents the time in years, with $t=3$ corresponding to 2003.\\
\im{94.png}\\
\begin{enumerate}[(a)]
    \item Use the regression capabilities of a graphing utility to fit
    linear and quadratic models to the data. Plot the data and
    graph the models.
        \indent\begin{align}
            \text{Linear model: }V&=-1686.8t+27501\\
            \text{Quadratic model: }V&=109.52t^2-3001.1t+31006
        \end{align}
    \item What does the slope represent in the linear model in
    part (a)?\\
        \indent The average value loss each year.
    \item Use the regression capabilities of a graphing utility to fit an
    exponential model to the data.\\
        \indent $V=30582.68(0.90724)^t=30582.68e^{-0.09735t}$
    \item Determine the horizontal asymptote of the exponential
    model found in part (c). Interpret its meaning in the context
    of the problem.\\
        \indent As $t\rightarrow 0$, $V\rightarrow 0$ in the model, indicating that the value tends to zero.
    \item Find the rate of decrease in the value of the sedan when $t=4$ and $t=8$ using the exponential model.
        \indent\begin{align}
            \set
            \text{When }t&=4,\,\, V'\approx -2017\,\,\text{dollars/year}\\
            \text{When }t&=8,\,\, V'\approx -1366\,\,\text{dollars/year}
        \end{align}
\end{enumerate}

\s{Find the indefinite integral.}
100.\begin{align}
    \set
    &\int e^{-x^4}(-4x^3)dx,\,\, u=-x^4,\,\, du=-4x^3dx\\
    &=e^{-x^4}+C
\end{align}

\vs\next
104.\begin{align}
    \set
    &\int e^x(e^x+1)^2dx,\,\, u=e^x+1,\,\, du=e^xdx\\
    &=\int(e^x+1)^2(e^x)dx\\
    &=\frac{(e^x+1)^3}{3}+C
\end{align}

\vs\next
108.\begin{align}
    \set
    &\frac{e^{2x}}{1+e^{2x}}dx,\,\, u=1+e^{2x},\,\, du=2e^{2x}dx\\
    &=\frac{1}{2}\int\frac{2e^{2x}}{1+e^{2x}}dx\\
    &=\frac{1}{2}\ln(1+e^{2x})+C
\end{align}

\vs\next
112.\begin{align}
    \set
    &\int\frac{2e^x-2e^{-x}}{(e^x+e^{-x})^2}dx\\
    &=2\int(e^x+e^{-x})^{-2}(e^x-e^{-2})dx\\
    &=\frac{-2}{e^x+e^{-x}}+C
\end{align}

\vs\next
116.\begin{align}
    \set
    &\int\ln(e^{2x-1})dx\\
    &=\int(2x-1)dx\\
    &=x^2-x+C
\end{align}

\s{Evaluate the definite integral.}
118.\begin{align}
    \set
    &\int_{3}^{4}e^{3-x}dx\\
    &=[-e^{3-x}]_3^4\\
    &=-e^{-1}+1\\
    &=1-\frac{1}{e}
\end{align}

\vs\next
122.\begin{align}
    \set
    &\int_{0}^{\sqrt[]{2}}xe^{-x^2/2}dx,\,\, u=\frac{-x^2}{2},\,\, du=-xdx\\
    &=-\int_{0}^{\sqrt[]{2}}e^{-x^2/2}(-x)dx\\
    &=[-e^{-x^2/2}]_0^{\sqrt[]{2}}\\
    &=1-e^{-1}
\end{align}

126.\begin{align}
    \set
    &\int_{\pi/3}^{\pi/2}e^{\sec 2x}\sec 2x\tan 2xdx,\,\, u=\sec 2x,\,\, du=2\sec 2x\tan 2xdx\\
    &=\frac{1}{2}\int_{\pi/3}^{\pi/2}e^{\sec 2x}(2\sec 2x\tan 2x)dx\\
    &=\frac{1}{2}[e^{\sec 2x}]_{\pi/3}^{\pi/2}\\
    &=\frac{1}{2}(e^{-1}-e^{-2})\\
    &=\frac{1}{2}\lt(\frac{1}{e}-\frac{1}{e^2}\rt)\\
    &=\frac{e-1}{2e^2}
\end{align}

\s{Solve the differential equation.}
128.\begin{align}
    \set
    \frac{dy}{dx}&=(e^x-e^{-x})^2\\
    y&=\int(e^x-e^{-x})^2dx\\
    &=\int(e^{2x}-2+e^{-2x})dx\\
    &=\frac{1}{2}e^{2x}-2x-\frac{1}{2}e^{-2x}+C
\end{align}

\s{Find the
particular solution that satisfies the initial conditions.}
130.\begin{align}
    \set
    f''(x)&=\sin x+e^{2x}\\
    f(0)&=\frac{1}{4}\\
    f'(0)&=\frac{1}{2}\\
    f'(x)&=\int(\sin x+e^{2x})dx=-\cos x+\frac{1}{2}e^{2x}+C_1\\
    f'(0)&=-1+\frac{1}{2}+C_1=\frac{1}{2}\therefore C_1=1\\
    f'(x)&=-\cos x+\frac{1}{2}e^{2x}+1\\
    f(x)&=\int(-\cos x+\frac{1}{2}e^{2x}+1)dx\\
    &=-\sin x+\frac{1}{4}e^{2x}+x+C_2\\
    f(0)&=\frac{1}{4}+C_2=\frac{1}{4}\therefore C_2=0\\
    f(x)&=x-\sin x+\frac{1}{4}e^{2x}
\end{align}

\s{Find the area of the region bounded
by the graphs of the equations.}
134.\begin{align}
    \set
    y&=e^{-x},\,\, y=0,\,\, x=a,\,\, x=b\\
    \int_{a}^{b}e^{-x}dx&=[-e^{-x}]_a^b=e^a-e^b
\end{align}

\vs\next
136.\begin{align}
    \set
    y&=e^{-2x}+2x,\,\, y=0,\,\, x=0,\,\, x=2\\
    \int_{0}^{2}(e^{-2x}+2)dx&=\lt[-\frac{1}{2}e^{-2x}+2x\rt]_0^2\\
    &=-\frac{1}{2}e^{-4}+4+\frac{1}{2}\approx 4.491
\end{align}

\s{Approximate
the integral using the Midpoint Rule, the Trapezoidal Rule, and
Simpson’s Rule with $n=12$.}
138.\begin{align}
    \set
    &\int_{0}^{2}2xe^{-x}dx,\,\, n=12\\
    &\text{Midpoint Rule = }1.1906\\
    &\text{Trapezoidal Rule = }1.1827\\
    &\text{Simpson's Rule = }1.1880\\
    &\text{Calculator = }1.18799
\end{align}

\s{Word problems}
140. The median waiting time (in minutes) for people
waiting for service in a convenience store is given by the
solution of the equation $\int_{0}^{x}0.3e^{-0.3}dt=\frac{1}{2}$. Solve the equation.
\begin{align}
    \set
    \int_{0}^{x}0.3e^{-0.3}dt=\frac{1}{2}&=\frac{1}{2}\\
    [-e^{-0.3t}]_0^x&=\frac{1}{2}\\
    -e{-0.3x}+1&=\frac{1}{2}\\
    e^{-0.3x}&=\frac{1}{2}\\
    -0.3x&=\ln\frac{1}{2}=-\ln 2\\
    x&=\frac{\ln 2}{0.3}\approx 2.31\text{ minutes}
\end{align}

\s{Capstone}
148. Describe the relationship between the graphs of $f(x)=\ln x$ and $g(x)=e^x$.\\
\indent Both graphs mirror each other across the line $y=x$.




























\end{document}