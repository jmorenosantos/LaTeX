\documentclass[11pt]{article}
\usepackage{amsmath, amssymb, amsfonts,  graphicx, enumerate, float, wrapfig, hyperref}
\usepackage[margin=0.5in]{geometry}
\graphicspath{{./}}
\newcommand*{\vs}{\vspace{1cm}}
\newcommand*{\next}{\noindent}
\newcommand*{\set}{\setcounter{equation}{0}}
\newcommand*{\im}{\includegraphics}
\newcommand*{\lt}{\left}
\newcommand*{\rt}{\right}

\begin{document}

\title{5.2 lab calculations}
\author{Juan J. Moreno Santos}
\date{December 2023}

\maketitle
\section{Dominance - D}
\begin{enumerate}
    \item Teacher lot:
        \begin{align}
            \set
            D_A=\frac{6}{54}
        \end{align}
    \item Student lot:
        \begin{align}
            \set
            D_B=\frac{6}{66}
        \end{align}
\end{enumerate}

\section{Species richness - R}
\[\text{Number of types of vehicles}-\text{Types of cars not in the parking lot}\]

\begin{enumerate}
    \item Teacher lot:
        \begin{align}
            \set
            R=41-15=26
        \end{align}
    \item Student lot:
        \begin{align}
            \set
            R=41-8=33
        \end{align}
\end{enumerate}

\section{Shannon-Wiener Index - H}
\begin{enumerate}
\item Teacher lot (Honda SUV):
\begin{align}
    \set
    H&=\sum(P_i\ln[P_i])\\
    P_i&=5\\
    N&=54\\
    P_i=\frac{n_i}{N}&=\frac{5}{54}\\
    H&=-\sum_{i=1}^{26}\frac{5}{54}\ln\frac{5}{54}=5.7
\end{align}
\item Student lot (Subaru car):
\begin{align}
    \set
    H&=-\sum(P_i\ln[P_i])\\
    P_i&=5\\
    N&=66\\
    P_i=\frac{n_i}{N}&=\frac{5}{66}\\
    H&=-\sum_{i=1}^{33}\frac{5}{66}\ln\frac{5}{66}=6.5
\end{align}
\end{enumerate}

\section{Species Evenness - E}
\begin{enumerate}
    \item Teacher lot:
        \begin{align}
            \set
            E&=\frac{H}{\ln(R)}\\
            &=\frac{5.7}{\ln(26)}=1.75
        \end{align}
    \item Student lot:
        \begin{align}
            \set
            E&=\frac{H}{\ln(R)}\\
            &=\frac{6.5}{\ln(33)}=1.86
        \end{align}
\end{enumerate}

\begin{enumerate}[(a)]
    \item 
        \begin{align}
            \set
            1950:\frac{5.2\cdot 10^{10}\text{kg}}{2.6\cdot 10^9\text{people}}=\text{20 meat kg per capita}\\
            2000:\frac{2.4\cdot 10^{11}\text{kg}}{6\cdot 10^9\text{people}}=\text{40 meat kg per capita}\\
            \frac{40\text{kg}}{20\text{kg}}=2\Rightarrow\text{100\% increase}
        \end{align}
\end{enumerate}









\end{document}