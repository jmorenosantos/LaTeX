\documentclass[11pt]{article}
\usepackage{amsmath, amssymb, amsfonts,  graphicx, enumerate, float, wrapfig, hyperref}
\usepackage[margin=0.5in]{geometry}
\graphicspath{{./}}
\newcommand*{\vs}{\vspace{1cm}}
\newcommand*{\next}{\noindent}
\newcommand*{\set}{\setcounter{equation}{0}}

\begin{document}

\next
(a) To identify the two countries with the same crude growth rate (r), we employ the formula:
\[r=\frac{\text{Crude Birth rate} - \text{Crude Death Rate} + \text{Immigration Rate}}{10}\]
Calculating $r$ for each country yields the following:
\begin{enumerate}[I.]
    \item \[r=\frac{15-13+3}{10}=0.5\%\]
    \item \[r=\frac{18-5+1}{10}=1.4\%\]
    \item \[r=\frac{20-10+1}{10}=1.1\%\]
    \item \[r=\frac{16-12+1}{10}=0.5\%\]
\end{enumerate}
Countries I and IV have the same $r$ of $0.5\%$.\\

\vs\next
(b) The doubling time $(t_d)$ for Country II is given by the Rule of 70:
\[t_d=\frac{70}{r}=\frac{70}{1.4}=50\,\,\text{years}\]

\next
(c) To predict a country's future population $(P_t)$, we consider $P_0$ as the curent population and use the formula
\[P_t=P_0e^{rt}\]
For each country:
\begin{enumerate}[I.]
    \item \[P_t=10000000\cdot e^{0.005\cdot 140}\approx 16621985\]
    \item \[P_t=25000000\cdot e^{0.014\cdot 140}\approx 314880594\]
    \item \[P_t=25000000\cdot e^{0.011\cdot 140}\approx 224882933\]
    \item \[P_t=100000000\cdot e^{0.005\cdot 140}\approx 166626794\]
\end{enumerate}

Conutry III is predicted to have the population closest to 200000000. Thus, Country III is the predicted country.
\end{document}
