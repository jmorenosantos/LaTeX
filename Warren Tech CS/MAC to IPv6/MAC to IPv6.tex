\documentclass[11pt]{article}
\usepackage{amsmath, amssymb, amsfonts,  graphicx, enumerate, float, wrapfig, hyperref}
\usepackage[margin=0.5in]{geometry}
\graphicspath{{./}}
\newcommand*{\vs}{\vspace{1cm}}
\newcommand*{\next}{\noindent}
\newcommand*{\set}{\setcounter{equation}{0}}

\begin{document}

\title{Convert Computer MAC Addresses to IPv6 EUI-64 Addresses}
\author{Juan J. Moreno Santos}
\date{December 7 2023}

\maketitle
\section{Using the six (fake) MAC addresses below, convert each from a 48-bit MAC address to a 64-bit (EUI-64) address.}
1. C9:2C:BF:7A:D5:DD\\
\indent In detail, the steps implemented in each exercise are the following:
\begin{enumerate}[i.]
    \item Split the MAC address into two 3-byte (24-bit) halves:
        \begin{align}
            \set
            \text{C9:2C:BF:7A:D5:DD} =\text{C9:2C:BF and 7A:D5:DD}
        \end{align}
    \item Add FFFE, the missing 16 bits, in the middle:
        \begin{align}
            \text{C9:2C:BF:FF:FE:7A:D5:DD}
        \end{align}
    \item Format in IPv6 notation, dividing the address into four hexadecimal bits:
        \begin{align}
            \text{C92C:BFFF:FE7A:D5DD}
        \end{align}
    \item Convert the first octet (C9 in this case) from hexadecimal to binary:
        \begin{align}
            \text{C9}=1100\,\,\text{and}\,\, 1001=11001001
        \end{align}
    \item Invert the seventh bit:
        \begin{align}
            11001001\Rightarrow 11001011
        \end{align}
    \item Convert the octet from binary back into hexadecimal:
        \begin{align}
            11001011=CB
        \end{align}
    \item Replace the first octet with the converted one:
        \begin{align}
            \text{C92C:BFFF:FE7A:D5DD}\Rightarrow \text{CB2C:BFFF:FE7A:D5DD}
        \end{align}
    \item Add fe80:: to the beginning of the address, and write it in lowercase:
        \begin{align}
            \lowercase\text{FE80::CB2C:BFFF:FE7A:D5DD}
        \end{align}
\end{enumerate}

\newpage\next
2. 08:9C:FD:C3:39:ED
\begin{align}
    \set
    &\text{08:9C:FD:C3:39:ED = 08:9C:FD and C3:39:ED}\\
    &\text{08:9C:FD:FF:FE:C3:39:ED}\\
    &\text{089C:FDFF:FEC3:39ED}\\
    &08=00001000\\
    &00001000\Rightarrow 00001010\\
    &00001010=\text{0A}\\
    &\text{089C:FDFF:FEC3:39ED}\Rightarrow\text{0A9C:FDFF:FEC3:39ED}\\
    &\lowercase\text{FE80::0A9C:FDFF:FEC3:39ED}
\end{align}

\vs\next
3. 5E:9D:8F:CF:3A:FE
\begin{align}
    \set
    &\text{5E:9D:8F:CF:3A:FE = 5E:9D:8F and CF:3A:FE}\\
    &\text{5E:9D:8F:FF:FE:CF:3A:FE}\\
    &\text{5E9D:8FFF:FECF:3AFE}\\
    &\text{5E}=01011110\\
    &01011110\Rightarrow 1011100\\
    &1011100=\text{5C}\\
    &\text{5E9D:8FFF:FECF:3AFE}\Rightarrow\text{5C9D:8FFF:FECF:3AFE}\\
    &\lowercase\text{FE80::5C9D:8FFF:FECF:3AFE}
\end{align}

\vs\next
4. BB:65:FD:22:FB:68
\begin{align}
    \set
    &\text{BB:65:FD:22:FB:68 = BB:65:FD and 22:FB:68}\\
    &\text{BB:65:FD:FF:FE22:FB:68}\\
    &\text{BB65:FDFF:FE22:FB68}\\
    &\text{BB}=10111011\\
    &10111011\Rightarrow 10111001\\
    &10111001=\text{B9}\\
    &\text{BB65:FDFF:FE22:FB68}\Rightarrow\text{B965:FDFF:FE22:FB68}\\
    &\lowercase\text{FE80::B965:FDFF:FE22:FB68}
\end{align}

\vs\next
5. 96:41:FD:C3:C1:C5
\begin{align}
    \set
    &\text{96:41:FD:C3:C1:C5 = 96:41:FD and C3:C1:C5}\\
    &\text{96:41:FD:FF:FE:C3:C1:C5}\\
    &\text{9641:FDFF:FEC3:C1C5}\\
    &\text{96}=10010110\\
    &10010110\Rightarrow 10010100\\
    &10010100=\text{94}\\
    &\text{9641:FDFF:FEC3:C1C5}\Rightarrow\text{9441:FDFF:FEC3:C1C5}\\
    &\lowercase\text{FE80::9441:FDFF:FEC3:C1C5}
\end{align}

\vs\next
6. AB:79:B0:CA:F7:BF
\begin{align}
    \set
    &\text{AB:79:B0:CA:F7:BF = AB:79:B0 and CA:F7:BF}\\
    &\text{AB:79:B0:FF:FE:CA:F7:BF}\\
    &\text{AB79:B0FF:FECA:F7BF}\\
    &\text{AB}=10101011\\
    &10101011\Rightarrow 10101001\\
    &10101001=\text{A9}\\
    &\text{AB79:B0FF:FECA:F7BF}\Rightarrow\text{A979:B0FF:FECA:F7BF}\\
    &\lowercase\text{FE80::A979:B0FF:FECA:F7BF}
\end{align}






























\end{document}